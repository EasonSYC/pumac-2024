\documentclass[11pt]{article}
\usepackage[utf8]{inputenc}
\usepackage{geometry}
\geometry{a4paper}

\usepackage{graphicx}
\usepackage{amsmath,amsthm,amssymb}
\usepackage{amsfonts}
\usepackage{eucal}
\usepackage{gensymb}
\usepackage{yfonts}
\usepackage{epic}
\usepackage{verbatim}
\usepackage{xcolor}
\usepackage{enumitem}
\usepackage{hyperref}
\usepackage{cleveref}
\hypersetup{colorlinks=true,citecolor=blue,linkcolor=blue}
\usepackage{textcomp}
\usepackage{marginnote}
\usepackage{color}
\usepackage{tikz}
\usepackage[font=small,labelfont=bf]{caption}
\usepackage[super]{nth}
\usepackage[framemethod=TikZ]{mdframed}
\mdfdefinestyle{MyFrame}{%
    linecolor=black!75!white,
    outerlinewidth=2pt,
    roundcorner=20pt,
    innertopmargin=\baselineskip,
    innerbottommargin=\baselineskip,
    innerrightmargin=20pt,
    innerleftmargin=20pt,
    backgroundcolor=white}
\definecolor{g3}{gray}{0.9}
\mdfdefinestyle{prob}{%
    linecolor=black!75!white,
    outerlinewidth=2pt,
    roundcorner=5pt,
    %innertopmargin=\baselineskip,
    %innerbottommargin=\baselineskip,
    innerrightmargin=20pt,
    innerleftmargin=20pt,
    backgroundcolor=g3}

\usepackage{fancyhdr}
\usepackage{float}
\usepackage{mdframed}

\usepackage{wrapfig}

\mdfsetup{nobreak=true}

%\oddsidemargin=.2in
%\evensidemargin=.2in
%\topmargin=-.5in
%\textheight=8.5in
%\textwidth=6in

\oddsidemargin=.2in
\evensidemargin=.2in
\topmargin=-.5in
\textheight=8.5in
\textwidth=6in

\usepackage{mathtools}
\DeclarePairedDelimiter{\ceil}{\lceil}{\rceil}

\fancypagestyle{firststyle}{
\fancyhead[L]{\includegraphics[height=35pt, width= 202 pt, keepaspectratio]{PUMaC_logo_color.png}}
\fancyhead[R]{\includegraphics[height=50pt, width=40 pt]{PU_Shield.png}}}

\fancypagestyle{plain}{
\lhead{PUMaC 2020 Power Round}
\lhead{\includegraphics[height=17.5pt, width= 101 pt, keepaspectratio]{PUMaC_logo_color.png}}
\rhead{\includegraphics[height=20pt, width=20 pt]{PU_Shield.png}}
\cfoot{}

\crefname{thrm}{}{Theorem}
\crefname{defn}{}{Definiton}

\renewcommand{\headrulewidth}{0.2pt}}
\newcommand{\half}{\frac{1}{2}}
\newcommand{\bbpr}[2]{\mathbb{#1}[#2]}
\newcommand{\E}{\mathbb{E}}
\newcommand{\tr}{\text{tr}}
\newcommand{\aaa}{\text{\textbf{a}}}
\newcommand{\bb}{\text{\textbf{b}}}
\newcommand{\cc}{\text{\textbf{c}}}
\newcommand{\ee}{\text{\textbf{e}}}
\newcommand{\hh}{\text{\textbf{h}}}
\newcommand{\ii}{\text{\textbf{i}}}
\newcommand{\jj}{\text{\textbf{j}}}
\newcommand{\mm}{\text{\textbf{m}}}
\newcommand{\vv}{\text{\textbf{v}}}
\newcommand{\ww}{\text{\textbf{w}}}
\newcommand{\xx}{\text{\textbf{x}}}
\newcommand{\zz}{\text{\textbf{z}}}
\newcommand{\yy}{\text{\textbf{y}}}
\newcommand{\Span}{\text{span}\,}
\newcommand{\C}{\mathbb{C}}
\newcommand{\CC}{\mathcal{C}}
\newcommand{\N}{\mathbb{N}}
\newcommand{\R}{\mathbb{R}}
\newcommand{\Q}{\mathbb{Q}}
\newcommand{\Z}{\mathbb{Z}}
\newcommand{\p}{\mathcal{P}}
\newcommand{\del}{\Delta}
\newcommand{\eps}{\varepsilon}
\newcommand{\sip}[2]{\left\langle#1,#2\right\rangle}
\newcommand{\abs}[1]{\left\lvert#1\right\rvert}
\newcommand{\rhat}{\hat{r}}
\newcommand{\inv}{^{-1}}
\newcommand{\nye}[1]{\tilde{#1}}
\newcommand{\gl}{\mathfrak{gl}}
\newcommand{\Sl}{\mathfrak{sl}}
\newcommand{\Sp}{\mathfrak{sp}}
\newcommand{\So}{\mathfrak{so}}
\newcommand{\ad}{\text{ad}\,}
\newcommand{\h}{\mathcal{H}}


\newcommand*\sierpinski{\vcenter{\hbox{\includegraphics[width=1em]{SierpinskiTriangleIcon.png}}}}



\DeclareMathOperator{\im}{im}

\newtheorem{theorem}[subsubsection]{Theorem}
\newtheorem{lemma}[subsubsection]{Lemma}
\newtheorem{prop}[subsubsection]{Proposition}
\newtheorem{cor}[subsubsection]{Corollary}
\theoremstyle{definition}
\newtheorem{defn}[subsubsection]{Definition}
\newtheorem*{example}{Example}
\theoremstyle{remark}
\newtheorem*{rmk}{Remark}

\newtheoremstyle{problem}{-\topsep}{}{\normalfont}{}{\bfseries}{.}{.5em}{}
\theoremstyle{problem}
\newmdtheoremenv{prob}{Problem}[subsection]
% \newenvironment{prob}{\begin{mdframed}\begin{pro}}
%   {\end{pro}\end{mdframed}}


\title{PUMaC 2024 Power Round: \\ Measures and Fractals}
\author{Colby Riley}
\date{Fall 2024}

\setcounter{section}{0}

\DeclareMathOperator{\Var}{Var}
\DeclareMathOperator{\ex}{ex}

\begin{document}


\thispagestyle{empty}
\noindent \huge{Team Number:} \underline{213\phantom{}}

\vspace{.5cm}
\noindent \huge{PUMaC 2024 Power Round Cover Sheet}

\vspace{.5cm}
\normalsize
In years past, this page was for hand-turned in submissions. Now it is for reference for you guys! So that you know how many points each problem is worth. 

\begin{center}
\begin{tabular}{|c|c|c|}\hline
Problem Number & Points & Attempted?\\\hline
2.0.1 & 5 & Yes \\\hline
2.0.2 & 5 & Yes \\\hline
2.0.3 & 5 & Yes \\\hline
2.0.4 & 10 & Yes \\\hline
2.0.5 & 5 & Yes \\\hline
2.1.1 & 15 & Yes \\\hline
2.1.2 & 5 & Yes \\\hline
2.1.3 & 10 & Yes \\\hline
2.1.4 & 15 & \\\hline
2.1.5 & 10 & \\\hline
2.1.6 & 5 & \\\hline
2.1.7 & 10 & \\\hline
2.1.8 & 10 & \\\hline
2.1.9 & 5 & \\\hline
4.0.1 & 15 & Yes \\\hline
4.0.2 & 10 & Yes \\\hline
4.0.3 & 15 & Yes \\\hline
4.0.4 & 10 & Yes \\\hline
4.1.1 & 5 & Yes \\\hline
4.1.2 & 5 & Yes \\\hline
4.1.3 & 5 & \\\hline
4.1.4 & 15 & \\\hline
4.1.5 & 10 & \\\hline
\end{tabular}
\hspace{0.5 cm}
\begin{tabular}{|c|c|c|}\hline
Problem Number & Points & Attempted?\\\hline
4.2.1 & 10 & \\\hline
4.2.2 & 10 & \\\hline
4.2.3 & 5 & \\\hline
4.2.4 & 5 & \\\hline
4.2.5 & 20 & \\\hline
4.2.6 & 15 & \\\hline
5.0.1 & 10 & \\\hline
5.0.2 & 5 & \\\hline
5.0.3 & 5 & \\\hline
5.0.4 & 10 & \\\hline
5.0.5 & 25 & \\\hline
5.0.6 & 40 & \\\hline
5.0.7 & 50 & \\\hline
\end{tabular}
\end{center}

\newpage
\tableofcontents
\newpage

\section{Problem 2.0.1}
\textit{Proof:} \begin{enumerate}
    \item This is a \(\sigma\)-algebra, because
    \begin{enumerate}
        \item \(\emptyset \in \Sigma\),
        \item \(X \setminus \emptyset = X \in \Sigma, X \setminus X = \emptyset \in \Sigma\),
        \item \(X \cup \emptyset = X \in \Sigma\),
        \item \(X \cap \emptyset = \emptyset \in \Sigma\).
    \end{enumerate}
    \item This is \textbf{not} a \(\sigma\)-algebra, because \(1 \in \Sigma\), but \(1 \notin \p(X)\), which implies \(\Sigma \not\subset \p(X)\) since \(\p(X)\) only contains subsets of \(X\) as members, and \(1\) is not a subset of \(X\).
    \item This is a \(\sigma\)-algebra, because \(\{0, 1, 2, 3, 4\} = X\), \(\{0, 1\}\) and \(\{2, 3, 4\}\) are complements, \(\emptyset\) and \(X\) are complements. Therefore rules 1 and 2 are satisfied.
    Unions with \(X\) are \(X \in \Sigma\), \(\{0, 1\} \cup \{2, 3, 4\} = X \in \Sigma\). Intersections with \(\emptyset\) are \(\emptyset \in \Sigma\), and \(\{0, 1\} \cap \{2, 3, 4\} = \emptyset \in \Sigma\).
    \item This is \textbf{not} a \(\sigma\)-algebra, because \(\{1, 2\} \in \Sigma\), but \(X \setminus \{1, 2\} = \{0, 3, 4\} \notin \Sigma\).
\end{enumerate}
\newpage

\section{Problem 2.0.2}
\textit{Proof:} \begin{enumerate}
    \item \(\Sigma = \p(X)\) is a \(\sigma\)-algebra because:
    \begin{enumerate}
        \item \(\emptyset \in \Sigma = \p(X)\),
        \item \(X \setminus E_i \subset X \), so \(X \setminus E_i \in \p(X)\),
        \item \(\bigcup_{i = 1}^{+\infty}E_i \subset X\), so \(\bigcup_{i = 1}^{+\infty}E_i \in \p(X)\),
        \item \(\bigcap_{i = 1}^{+\infty}E_i \subset X\), so \(\bigcap_{i = 1}^{+\infty}E_i \in \p(X)\),
    \end{enumerate}
    In short, \(\Sigma = \p(X)\) is closed under complement, intersection and union, and therefore a \(\sigma\)-algebra.
    \item \(\mu\) is a measure because \(\mu: \Sigma \to \overline{\R}\) satisfies that:
    \begin{enumerate}
        \item \(\mu(\emptyset) = \# \emptyset = 0\),
        \item for all \(E \in \p(X), \mu(E) = \# E \geq 0\),
        \item let \(\{E_i\}_{i = 0}^{+\infty}\) be pairwise disjoint,
        \begin{align*}
            \mu\left(\bigcup_{i = 1}^{+\infty} E_i\right) &= \# \bigcup_{i = 1}^{+\infty} E_i\\
            &= \sum_{i = 1}^{+\infty} \# E_i\\
            &= \sum_{i = 1}^{+\infty} \mu(E_i),
        \end{align*}
        the second equal sign being true since all the sets are pairwise disjoint.
    \end{enumerate}
\end{enumerate}
\newpage

\section{Problem 2.0.3}
\textit{Proof:} Please add the solution to solutions/section-2-0/q-2-0-3.tex.
\newpage

\section{Problem 2.0.4}
\textit{Proof:} We define a series of sets \(\{D_i\}_{i = 1}^{+\infty}\) as follows:
\begin{itemize}
    \item \(D_1 = E_1\),
    \item \(D_i = E_i \setminus \bigcup_{k = 1}^{i - 1} E_k\).
\end{itemize}

Clearly \(D_i \subset E_i\). Let \(i \neq j\), and W.L.O.G. let \(i < j\), where \(i, j \in \N\). We have
\[
D_i \subset E_i \subset \bigcup_{k = 1}^{j - 1}E_k, D_j = E_j \setminus \bigcup_{k = 1}^{j - 1}E_k,
\]
which implies \(D_i \cap D_j = \emptyset\). This means that \(\{D_i\}_{i = 1}^{+\infty}\) are pairwise disjoint.

Let \(F = \bigcup_{i = 1}^{+\infty}D_i\). We claim that
\[
F = \bigcup_{i = 1}^{+\infty}E_i.
\]

\begin{itemize}
    \item \(\bigcup_{i = 1}^{+\infty}E_i \subset F\). For any \(a \in \bigcup_{i = 1}^{\infty} E_i\), we must have \(a \in E_j\) for some natural number \(j\). Let \(k\) be the minimal of such \(j\) by well-ordering principle, i.e.
    \[
    a \in E_k, a \notin \bigcup_{i = 1}^{k - 1} E_i.
    \]
    This implies that \(a \in E_k \setminus \bigcup_{i = 1}^{n - 1} E_i = D_k\), which therefore means \(a \in \bigcup_{i = 1}^{+\infty}D_i = F\), and therefore
    \[\bigcup_{i = 1}^{+\infty}E_i \subset F.\]

    \item \(F \subset \bigcup_{i = 1}^{+\infty}E_i\). For any \(b \in F = \bigcup_{i = 1}^{+\infty} D_i\), \(b\) must be a member of \(D_k\) for some \(k \in \mathbb{N}\).

    Since \(b \in D_k \subset E_k \subset \bigcup_{i = 1}^{+\infty}E_i\), this means that \(b \in F\), and therefore
    \[F \subset \bigcup_{i = 1}^{+\infty}E_i.\]
\end{itemize}

Now, we can show that
\begin{align*}
    \mu\left(\bigcup_{i = 1}^{+\infty} E_i\right) = \mu(F) &= \mu\left(\bigcup_{i = 1}^{+\infty} D_i\right)\\
    &= \sum_{i = 1}^{+\infty} \mu(D_i)\\
    &\leq \sum_{i = 1}^{\infty} \mu(E_i),
\end{align*}
as desired.

Note the equal sign on the second line arises because \(D_i\) are pairwise disjoint, and the less or equal sign on the third line arises because \(D_i \subset E_i\).
\newpage

\section{Problem 2.0.5}
\textit{Proof:} Yes, \((X, \Sigma, \mu)\) is a measure space.

We have verified before that \(\p(X)\) is a \(\sigma\)-algebra over \(X\). Now we verify that \(\mu\) is a measure for such \(\sigma\) algebra.
\begin{enumerate}
    \item \(\mu(\emptyset) = 0\) because \(x_0 \notin \emptyset\),
    \item \(\mu(E) = 0\) or \(1\), so \(\mu(E) \geq 0\) for every \(E \in \Sigma\),
    \item let \(\{E_i\}_{i = 1}^{+\infty}\) be a collection of pairwise disjoint sets.
    \begin{itemize}
        \item \textbf{Case 1.} There exists such \(k \in \N\), \(x_0 \in E_k\). Then by definition \(\mu(E_k) = 1\). Since \(E_i\) are disjoint, for all \(n \neq k, n \in \N\), we must have \(x_0 \notin E_n\), and therefore \(\mu(E_n) = 0\).

        Also, \(x_0 \in E_k \subset \bigcup_{i = 1}^{+\infty} E_i\), so
        \[
        \mu\left(\bigcup_{i = 1}^{+\infty} E_i\right) = 1 = \sum_{i = 1}^{+\infty} \mu(E_i),
        \]
        and the third axiom for measures holds in this case.
        
        \item \textbf{Case 2.} For all \(k \in \N\), \(x_0 \notin E_k\). Then by definition \(\mu(E_k) = 0\) for all \(k\), and the sum \(\sum_{i = 1}^{+\infty} \mu(E_i) = 0\).

        Also,we will have \(x_0 \notin \bigcup_{i = 1}^{+\infty} E_i\), which means
        \[
            \mu\left(\bigcup_{i = 1}^{+\infty} E_i\right) = 0 = \sum_{i = 1}^{+\infty} \mu(E_i),
        \]
        and the third axiom for measures holds in this case.
    \end{itemize}

    Therefore all axioms for the measure hold, so \(\mu\) is a measure on \(\Sigma\), and so \((X, \Sigma, \mu)\) is a measure space.
\end{enumerate}
\newpage

\section{Problem 2.1.1}
\textit{Proof:} Let \(E_0\) be the base set \(E\), and let 
\[E_{i+1} = E_i \cup \bigcup_{a, b \in E_i} \overline{ab}, E_\infty = \bigcup_{i=0}^{+\infty} E_i.\]

\(E_\infty\) convex: if \(a, b \in E_\infty\), then \(a \in E_i, b \in E_j\) for some \(i, j \in \N\), and therefore \(a, b \in E_{\max\{i, j\}}\) and so W.L.O.G. we simply denote this \(a, b \in E_i\) for some \(i \in \N\), then we will have \(\overline{ab} \in E_{i+1}\).

We first show that \(|E_{i+1}| \leq |E_i|\).
\begin{align*}
    |E_{i+1}| &= \sup_{p_1, p_2 \in E_{i+1}} |p_1-p_2| \\
    &\leq \sup_{\substack{a_1, a_2, b_1, b_2 \in E_i,\\p_1 \in \overline{a_1b_1},\\p_2 \in \overline{a_2b_2}}} |p_1-p_2| \\
    &\leq \sup_{\substack{a_1, a_2, b_1, b_2 \in E_i, \\ p_1 \in \overline{a_1b_1}}} \max(|p_1-a_2|, |p_1-b_2|) \\
    &\leq \sup_{a_1, a_2, b_1, b_2 \in E_i} \max(|a_1-a_2|, |b_1-a_2|, |a_1-b_2|, |b_1-b_2|) \\
    &\leq \sup_{x, y \in E_i} |x-y| \\
    &= |E_i|,
\end{align*}
where the first and second inequalities are due to the fact that for a triangle \(ABC\),
\[\sup_{P \in \overline{BC}} |A-P| \leq \max(|A-B|, |A-C|).\]

Hence \(|E_0| \leq |E_1| \leq \ldots \leq |E_\infty|\).

Now let
\[D = \bigcup_{p \in E_{\infty}} B_p\left(\frac{\epsilon}{2}\right).\]

\(D\) is open as it's a union of open balls.

By construction of \(D\), let \(a \in D\) be in \(B_{p_a}\left(\frac{\epsilon}{2}\right)\) for \(p_a \in E_\infty\).

We show that \(|D| \leq \epsilon + |E|\).
\begin{align*}
    |D| &= \sup_{a,b \in D} |a-b|\\
    &= \sup_{a,b \in D} |(a-p_a) + (p_a-p_b) + (p_b-b)|\\
    &\leq \sup_{a,b \in D} \left(|a-p_a| + |p_a-p_b| + |p_b-b|\right)\\
    &\leq \sup_{a,b \in D} \left(\frac{\epsilon}{2} + |p_a-p_b| + \frac{\epsilon}{2}\right)\\
    &\leq \epsilon + \sup_{a,b \in D} |p_a-p_b| \\
    &\leq \epsilon + \sup_{a,b \in E_\infty} |a-b| \\
    &\leq  \epsilon + |E_\infty| \\
    &\leq  \epsilon + |E_0| \\
    &=  \epsilon + |E|,
\end{align*}
as required. (The fourth inequality is from the fact that $p_i \in E_{\infty}$.)

Now we show \(D\) is convex. Let \(a, b \in D, p \in \overline{ab}\), let \(q\) be the closest point on \(\overline{p_ap_b}\) to \(p\). We have \(|p-q| \leq \max(|p_a - a|, |p_b - b|)\) by geometry, and \(q \in \overline{p_ap_b}\) by convexity of \(E_\infty\), so
\begin{align*}
    |p-q| &\leq \max(|p_a - a|, |p_b - b|) \\
    &\leq \frac{\epsilon}{2},
\end{align*}
and therefore \(p \in B_q\left(\frac{\epsilon}{2}\right) \subseteq D\) by construction of D. 
Thus all the properties of \(U = D\) are satisfied.





\newpage

\section{Problem 2.1.2}
\textit{Proof:} None of them are \(\delta\)-covers.

\begin{enumerate}
    \item \(\frac{1}{3}\) isn't in any of the intervals, but it's in \([0,1]\), and we want \([0,1] \subset \bigcap_i U_i\).
    \item We want \(|U_i| < \delta = \frac{1}{2}\) for all \(i\), but \(U_1 = [0,\frac{1}{1}] = [0,1]\), and
    \[|[0,1]| = \sup_{x,y \in [0,1]} |y-x| \geq |1-0| = 1 > \frac{1}{2}.\]
    \item We want \(|U_i| < \delta = \frac{1}{2}\) for all \(i\), but
    \[|[0,\frac{1}{2}]| = \sup_{x,y \in [0,\frac{1}{2}]} |y-x| \geq \left|\frac{1}{2} - 0\right| = \frac{1}{2} = \delta\]
    so it's invalid.
\end{enumerate}
\newpage

\section{Problem 2.1.3}
\textit{Proof:} Let \[M_\delta(F) = \{\{U_i\}_i : \{U_i\}_i \text{ is a } \delta\text{-cover of } F\}.\]

\textbf{Claim.} \(M_{\delta_1}(F) \subset M_{\delta_2}(F)\) for \(0 < \delta_1 < \delta_2\).

\textit{Proof.} Take \({\{U_i\}_i} \in M_{\delta_1(F)}\). Then \(\{U_i\}_i\) is a \(\delta_2\) cover of \(F\).

Therefore, \(F \subset \bigcup_i U_i\) and \(|U_i| < \delta_1\).

\(\delta_1 < \delta_2\) so \(|U_i| < \delta_1 < \delta_2\).

Thus \(\{U_i\}_i\) is a \(\delta_2\)-cover of \(F\), and thus \(\{U_i\}_i \in M_{\delta_2}(F)\). \qed

Let
\[R_\delta(F) = \left\{\sum_{i=1}^\infty |U_i|^s: \{U_i\}_i \in M_\delta(F)\right\}.\]

\begin{align*}
    M_{\delta_1}(F) \subset M_{\delta_2}(F) &\implies R_{\delta_1}(F) \subset R_{\delta_2}(F)\\
    &\implies \inf(R_{\delta_1}(F)) \geq \inf(R_{\delta_2}(F))\\
    &\implies \h_{\delta_1}^s(F) \geq \h_{\delta_2}^s(F)
\end{align*}
as desired.
\newpage

\section{Problem 2.1.4}
\textit{Proof:} 
\begin{enumerate}
    \item By definition,
    \[\h^s_\delta(E) = \inf \left\{\sum_{i=1}^\infty |U_i|^s : \{U_i\}\text{ is a }\delta\text{-cover of }E\right\}.\]

    \(\sum_{i=1}^\infty |U_i|^s \geq 0\) as \(|U_i| \geq 0\), so \(\h^s_\delta(E) \geq 0 \implies \h^s \geq 0\) (as if the limit approaches \(L < 0\), there must be infinitely many terms in the ball with radius \(-L\) around \(L\), but there can't be any as all this terms in this ball are less than 0, while  \(\h_\delta^s(E) \geq 0\) for all \(\delta\)).

\item By definition,
    \[\h^s_\delta(\emptyset) = \inf \left\{ \sum_{i=1}^\infty |U_i|^s : \{U_i\} \text{ is a }\delta\text{-cover of }\emptyset\right\}.\]
    
    But \(\emptyset\) is a \(\delta\)-cover as \(\emptyset \subset \bigcup_{S \in \emptyset} S = \emptyset\), and trivially \(|U_i| < \delta\) for all \(U_i \in \emptyset\) (as there are none).

    \(\sum_i |U_i|^s = 0\) as this is an empty sum, hence by definition of the \(\inf\) operator, \(\h_\delta^s (\emptyset) \leq 0\). By part 1, \(\h^s_\delta(E) \geq 0\) so \(\h_\delta^s (\emptyset) = 0\).
    
    Hence \(\h^s (\emptyset) = 0\) as the limit of a sequence which is identically 0 is 0 (for any \(\epsilon\)-ball around \(0\) all the terms are always in this ball).

\item  Consider a countable set of $\{E_j\}_{j = 0}^{+\infty}$;

    \begin{align*}
    \sum_{j = 1}^{+\infty} \h_d^s(E_j) &= \sum_{j = 1}^{+\infty} \inf \left\{ \sum_{i = 1}^{+\infty} |(U_j)_i|^s : (U_j)_i \text{ is a }\delta\text{-cover of } E_j \right\} \\
    &= \inf \left\{ \sum_{j = 1}^{+\infty} \sum_{i = 1}^{+\infty} |(U_j)_i|^s : (U_j)_i \text{ is a }\delta\text{-cover of } E_j \right\}. 
    \end{align*}
    Now, \(\bigcup_{j = 1}^{+\infty} \{(U_j)_i\}_i\) is a \(\delta\)-cover of \(\bigcup_{j = 1}^{+\infty} E_j\) (where \(\{(U_j)_i\}_{i = 1}^{+\infty} \text{ is a }\delta\text{-cover of } E_j\)): each \(\{(U_j)_i\}_i\) has elements with diameter \(< \delta\) so their union does as well, and each point in \(\bigcup_{j = 1}^{+\infty} E_j\) is in one of the \(E_j\)s so is covered in one of the \((U_j)_i\)s (as they are covers of \(E_j\)). 
    
    Thus, 
    \[ \left\{ \sum_{i,j} |(U_j)_i|^s : (U_j)_i \text{ is any }\delta\text{-cover of } E_j \right\} \geq  \sum_{k} |U_k|^s: U_k  \text{ is some }\delta\text{-cover of } \bigcup_j E_j;\]
    by problem 2.0.3;
    
    Hence
    \begin{align*}
        \inf \left\{ \sum_{i,j} |(U_j)_i|^s : (U_j)_i \text{ is a }\delta\text{-cover of } E_j \right\} &\geq \inf \left\{\sum_k |U_k|^s: U_k  \text{ is a }\delta\text{-cover of } \bigcup_j E_j\right\}\\
        &= \h_d^s\left(\bigcup_j E_j\right).
    \end{align*}
    
    Thus we have \(\sum_j \h_d^s(E_j) \geq  \h_d^s(\bigcup_j E_j)\) for all \(\delta > 0\), so it also holds in the limit as \(\delta \to 0\).

\end{enumerate}

\newpage

\section{Problem 2.1.5}
\textit{Proof:} For \(\delta > 0\) and \(A \subset \R^n\), let
\[
M_\delta(A) = \{\{U_i\} : \{U_i\} \text{ is a \(\delta\)-cover of } A\}.
\]

For \(A \subset \R^n\), \(c \in \R\) and \(c > 0\). Define
\[
c A = \{cx : x \in A\}.
\]

Note that \(E' = cE\).

Note that
\begin{align*}
    |cA| &= \sup_{x', y' \in cA} |x' - y'|\\
    &= \sup_{x, y \in A} |cx - cy|\\
    &= \sup_{x, y \in A} c|x - y|\\
    &= c \sup_{x, y \in A} |x - y|\\
    &= cA.
\end{align*}

We claim that
\[\{U_i\} \in M_{\delta}(E) \iff \{c U_i\}\in M_{c\delta}(E').\]

\begin{itemize}
    \item \textbf{\(\implies\) direction.} Let \(\{U_i\} \in M_\delta(E)\). Therefore \(E \subset \bigcup_i U_i\) and \(|U_i| < \delta\).

    Consider \( \{c U_i\}\). For any \(x' \in E'\), there exists \(x \in E\) such that \(x' = cx\) by definition.

    \begin{align*}
        E \subset \bigcup_i U_i &\implies \text{There exists } j \in \N \text{ such that } x \in U_j\\
        &\implies x' = cx \in c U_j\\
        &\implies x' \in \{c U_i\}.
    \end{align*}

    Thus \(E' \subset \bigcup_i c U_i\).

    Also, \(|cU_i| = c|U_i| < c\delta\), and therefore \(|c U_i| < c\delta\), and \(E' \subset \bigcup_{i} c U_i\).

    This means \(\{c U_i\}\)  is a \(c\delta\)-cover of \(E'\), and therefore \( \{cU_i\} \in M_{c\delta}(E')\).

    \item \textbf{\(\impliedby\) direction.} By definition, we also have \(E = \frac{1}{c} E'\), where \(\frac{1}{c} \in \R\) and \(\frac{1}{c} > 0\).

    Therefore, by the \(\implies\) direction, we also have
    \[
        \{c U_i\}\ \in M_{c\delta} (E') \implies \left\{\frac{1}{c} \cdot c \cdot U_i\right\} \in M_{\frac{1}{c}\cdot c\delta}\left(\frac{1}{c}E'\right) \implies \{U_i\} \in M_\delta (E).
    \]
\end{itemize}

Therefore we  have \[\{U_i\} \in M_{\delta}(E) \iff \{c U_i\}\in M_{c\delta}(E')\].

Now, we have
\begin{align*}
    \h_{c\delta}^s (E') &= \inf \left\{\sum_{i = 1}^{+\infty} |c  U_i|^s : \{c U_i\} \in M_{c\delta} (E')\right\}\\
    &= \inf \left\{\sum_{i = 1}^{+\infty} \left(c \cdot |U_i|\right)^s : \{U_i\} \in M_{\delta} (E)\right\}\\
    &= \inf \left\{c^s \sum_{i = 1}^{+\infty} |U_i|^s : \{ U_i\} \in M_{\delta} (E)\right\}\\
    &= c^s \cdot \h_{\delta}^s (E).
\end{align*}

So for any \(\delta > 0\), we will have
\[
\h_{c\delta}^s(E') = c^s \h_{\delta}^s(E).
\]

Therefore,
\begin{align*}
    \h^s(E') &= \lim_{c\delta \to 0} \h_{c\delta}^s(E')\\
    &= \lim_{c\delta \to 0} c^s \h_{\delta}^s(E)\\
    &= c^s \lim_{\delta \to 0} \h_{delta}^s(E)\\
    &= c^s \h^s(E),
\end{align*}
where \(\lim_{c\delta \to 0}\) is the same as \(\lim_{\delta \to 0}\) since \(c\) is a positive constant.
\newpage

\section{Problem 2.1.6}
\textit{Proof:} For \(\delta > 0\), \(A \subset \R^n\), let
\[
M_\delta(A) = \{\{U_i\} : \{U_i\} \text{ is a } \delta\text{-cover of } A\}.
\]

For any \(A \subset \R^n, x \in \R^n\), we define \(A + x = \{a + x: a \in A\}\).

Note that
\begin{align*}
    |A + x| &= \sup_{y', z' \in A + x} |y' - z'|\\
    &= \sup_{y, z \in A} |(y + x) - (z + x)|\\
    &= \sup_{y, z \in A} |y - z|\\
    &= |A|.
\end{align*}

\textbf{Claim.}
\[
\{U_i\} \in M_{\delta}(E) \iff \{U_i + x\} \in M_{\delta}(E').
\]

\textit{Proof.}
\begin{itemize}
    \item \textbf{\(\implies\) direction.} Let \(\{U_i\} \in M_{\delta}(E)\), which implies \(E \subset \bigcup_i U_i\) and \(|U_i| < \delta\).

    Consider \(\{U_i + x\}\). We have \(|U_i + x| = |U_i| < \delta\). Also, let \(y' \in E'\). Then there must exist some \(y \in E\) such that \(y' = y + x\).
    \begin{align*}
        y \in E \subset \bigcup_i U_i &\implies y \in U_j \text{ for some } j \in \N\\
        &\implies y' = y + x \in \left(U_j + x\right)\\
        &\implies y' \in \{U_j + x\}.
    \end{align*}
    So \(E' \subset \bigcup_i (U_i + x)\), and therefore \(\{U_i + x\}\) is a \(\delta\)-cover of \(E'\), and so \(\{U_i + x\} \in M_{\delta}(E')\).
    \item \textbf{\(\impliedby\) direction.} Note that \(E = E' + (-x)\) and \((-x) \in \R^n\). So therefore by the forward direction,
    \[
        \{U_i + x\} \in M_{\delta}(E') \implies \{U_i + x + (-x)\} \in M_{\delta} (E' + (-x)) \implies \{U_i\} \in M_{\delta}(E).
    \]
\end{itemize}
This finishes our proof of the claim. \qed

Now we have
\begin{align*}
    \h_{\delta}^s (E') &= \inf\left\{\sum_{i = 1}^{+\infty} |U_i + x|^s : \{U_i + x\} \in M_{\delta} (E')\right\}\\
    &= \inf\left\{\sum_{i = 1}^{+\infty} |U_i|^s : \{U_i\} \in M_{\delta} (E)\right\}\\
    &= \h_{\delta}^s (E).
\end{align*}

Since \[\h_{\delta}^s (E') = \h_{\delta}^s (E)\] is true for all \(\delta > 0\), we will have
\[\h^s(E') = \lim_{\delta \to 0} \h_{\delta}^s (E') = \lim_{\delta \to 0} \h_{\delta}^s (E) = \h^s(E).\]
\newpage

\section{Problem 2.1.7}
\textit{Proof:} \begin{enumerate}
    \item Let \(0 < \delta < 1\), and let \(\{U_i\}\) be a \(\delta\)-cover of \(E\).
    \begin{align*}
        &\phantom{\implies} 0 \leq |U_i| < \delta < 1, r < s\\
        &\implies |U_i|^s \leq |U_i|^r\\
        &\implies \sum_{i = 1}^{+\infty} |U_i|^s \leq \sum_{i = 1}^{+\infty} |U_i|^r.
    \end{align*}

    By definition,
    \begin{align*}
        &\phantom{\implies} \h_\delta^s(E) = \inf \left\{\sum_{i = 1}^{+\infty} |U_i|^s : \{U_i\} \text{ is a } \delta\text{-cover of } E\right\}\\
        &\implies \h_{\delta}^s(E) \leq \sum_{i = 1}^{+\infty} |U_i|^s \leq \sum_{i = 1}^{+\infty} |U_i|^r\text{ where } \{U_i\} \text{ is a } \delta\text{-cover of } E, \\
        &\implies \h_\delta^s(E) \leq \inf\left\{\sum_{i = 1}^{+\infty} |U_i|^r : \{U_i\} \text{ is a } \delta\text{-cover of } E\right\} = \h_\delta^r(E)
    \end{align*}

    \item
    \begin{align*}
        |U_i| \geq 0, s > 0 &\implies \sum_{i = 1}^{+\infty} |U_i|^s \geq 0\\
        &\implies 0 \leq \h_{\delta}^s(E) =  \inf \left\{\sum_{i = 1}^{+\infty} |U_i|^s : \{U_i\} \text{ is a } \delta\text{-cover of } E\right\}.
    \end{align*}

    So \(\h_{\delta}^s (E) \geq 0 \) for all \( \delta > 0\).

    We will show that \(\h_{\delta}^s(E)\) gets arbitrarily close to \(0\) as \(\delta \to 0\), which then implies \(\h^s(E) = 0\).

    Let \(b = s - r > 0\) since \(r < s\). Let \(0 < \delta < 1\). Let \(\{U_i\}\) be a \(\delta\)-coverage of \(F\), and therefore \(|U_i| < \delta < 1\).

    \begin{align*}
        &\phantom{\implies} |U_i|^s = |U_i|^r \cdot |U_i|^b\\
        &\implies |U_i|^s \leq |U_i|^r \delta^b\\
        &\implies \sum_{i = 1}^{+\infty} |U_i|^s \leq \sum_{i = 1}^{+\infty} |U_i|^r \delta^b = \delta^b \sum_{i = 1}^{+\infty} |U_i|^r.
    \end{align*}

    \begin{align*}
        &\phantom{\implies} \h_\delta^r(E) = \inf \left\{\sum_{i = 1}^{+\infty} |U_i|^r : \{U_i\} \text{ is a } \delta \text{-cover of } E\right\}\\
        &\implies \text{For all }\epsilon > 0, \text{ there exists a }\delta\text{-cover }\{U_i\} \text{ such that } \h_{\delta}^r(E) + \epsilon > \sum_{i = 1}^{+\infty} |U_i|^r\\
        &\implies \delta^b (\h_\delta^r(E) + \epsilon) > \delta^b \sum_{i = 1}^{+\infty} |U_i|^r \geq \sum_{i = 1}^{+\infty} |U_i|^s,
    \end{align*}
    and thus \(\h_\delta^s(E) = \inf \left\{\sum_{i = 1}^{+\infty} |U_i|^s : \{U_i\} \text{ is a } \delta \text{-cover of } E\right\} < \delta^b (\h_\delta^r(E) + \epsilon)\).

    Setting \(\epsilon = 1\), we get \(\h_\delta^s(E) < \delta^b (\h_\delta^r(E) + 1)\).

    From problem 2.1.3, we know that \(\h_\delta^r(E)\) weakly increases as \(\delta\) decreases towards 0, and therefore
    \[
    \h^r(E) = \lim_{\delta \to 0} \h_\delta^r(E) \geq \h_\delta^r(E),
    \]
    so for any \(\delta > 0\), we have \(\h_\delta^s(E) < \delta^b (\h^r(E) + 1)\).

    By our assumption, \(\h^r(E) + 1\) is fixed and \(< \infty\), so as \(\delta \to 0\), \(\delta^b(\h^r(E) + 1)\) gets arbitrarily close to 0. It is decreasing as  \(\delta \to 0\), therefore the limit is defined as  \(\inf( \lim_{\delta \to 0} \delta^b(\h^r(E) + 1))\), which is 0. Therefore,
    \[
    \h^s(E) = \lim_{\delta \to 0} \h_{\delta}^s (E) = 0,
    \]
   so \(\h^s(E) = 0\) as desired.
\end{enumerate}
\newpage

\section{Problem 2.1.8}
\textit{Proof:} Let \(E \subset \R^n, n \in \N\). Let \(A\) be the \(n\)-dimensional unit open hypercube centred at the origin, i.e., \(A = (-0.5, 0.5)^n\).

Let \(s > n\). We prove the following claim:

\textbf{Claim.} \(\h^s(A) = 0\).

\textit{Proof.} Let \(\delta > 0.\) We can partition \(A\) into closed \(n\)-dimensional hypercubes of side length \(\frac{\delta}{2\sqrt{n}}\).

It is clear that these cubes have diameter equal to the length of the diagonal that goes from \((0, 0, \ldots, 0)\) to \((1, 1, \ldots, 1)\). And therefore, by Pythagoras theorem, the diameter is \(\sqrt{\left(\frac{\delta}{2\sqrt{n}}\right)^2 \cdot n} = \frac{\delta}{2} < \delta\).

To cover one edge of \(A\), we need \(\frac{1}{\left(\frac{\delta}{2\sqrt{n}}\right)} = \frac{2\sqrt{n}}{\delta}\) cubes, so to cover all of \(A\), we will need \(\left(\frac{2\sqrt{n}}{\delta}\right)^n = \frac{\left(2\sqrt{n}\right)^n}{\delta^n}\) cubes.

Therefore,
\begin{align*}
    &\phantom{=} \sum_{i = 1}^{\frac{\left(2\sqrt{n}\right)^n}{\delta^n}} | \text{small cube}|^s\\
    &= \frac{\left(2\sqrt{n}\right)^n}{\delta^n} \cdot \left(\frac{\delta}{2}\right)^s\\
    &= \delta^{s - n} \cdot \left(\frac{\left(2\sqrt{n}\right)^n}{2^s}\right).
\end{align*}

Since these cubes cover \(A\), and each have diameter \(< \delta\), they are a \(\delta\)-cover of \(A\), and therefore, by definition,
\[
H_\delta^s(A) \leq \delta^{s - n}\left(\frac{\left(2\sqrt{n}\right)^n}{2^s}\right).
\]

\(s - n > 0\) since \(s > n\), and therefore, 
\[
\lim_{\delta \to 0} \delta^{s - n} \left(\frac{\left(2\sqrt{n}\right)^n}{2^s}\right) = 0 \implies \h^s(A) = \lim_{delta \to 0} \h_{\delta}^s(A) \leq 0.
\]

By part 1 of problem 2.1.4, we know \(\h^s(A) \geq 0\), and therefore \(\h^s(A) = 0\). \qed

Now consider \(cA = \{cx : x \in A\}\) for \(c > 0\). By problem 2.1.5, \(\h^s(cA) = c^s \h^s(A) = 0\). \(cA = \left(-\frac{c}{2}, \frac{c}{2}\right)^n\). Let
\[
B = \bigcup_{i \in \N} iA.
\]

\textbf{Claim.} \(B = \R^n\).

\textit{Proof.} \(A \subset \R^n \implies iA \subset \R^n \implies B \subset \R^n\).

Take some \(x \in \R^n\) and let \(x = (x_1, x_2, \ldots, x_n)\). Let
\[
m = \floor{\max |x_i|} + 1 \implies |x_i| < m \text{ for all } 1 \leq i \leq n.
\]

Consider \(2mA = (-m, m)^n\). Clearly \(x = \left(x_1, \ldots, x_n\right) \in 2mA\) which in turn implies that \(x \in B = \bigcup_{i \in \N} iA\), and therefore \(\R^n \subset B\).

Therefore \(B = \R^n\). \qed

By part 3 of problem 2.1.4, we must have
\[
\h^s (\R^n) \leq \sum_{i = 1}^{+\infty} \h^s(iA) = 0,
\]
but by part 1 of problem 2.1.4, \(\h^s (\R^n) \geq 0\). This means that \(\h^s(\R^n) = 0\).

By problem 2.0.3 and theorem 2.1.1, for \(E \subset \R^n\), we must have
\[
\h^s(E) \leq \h^s (\R_n) = 0,
\]
but also by part 1 of 2.1.4, we have \(\h^s(E) \geq 0 \implies \h^s(E) = 0\).

Therefore, for all \(s > n\), we have \(\h^s(E) = 0\).

B.W.O.C. assume that the Hausdorff dimension of \(E = d > n\). Therefore, for all \(r < d\), we havve \(\h^r(E) = +\infty\) by definition 2.1.2 and part 1 of problem 2.1.7.

Since \(d > n\), we may choose \(n < s < d\). \(s > n \implies \h^s(E) = 0\), but \(s < d \implies \h^s(E) = +\infty\), which is a contradiction.

Therefore, the Hausdorff dimension of \(E \leq n\), which finishes our proof of the question.
\newpage

\section{Problem 2.1.9}
\textit{Proof:} Let \(E \in \R^n\), we consider two cases on \(E\).
\begin{itemize}
    \item \textbf{\(E\) is finite}. Let \(x_1, \ldots, x_m\) be the elements of \(E\), where \(x_i \neq x_j\) for \(i \neq j\), i.e.,
    \[
    E = \{x_1, x_2, \ldots, x_m\},
    \]
    where \[\# E = m.\]

    Let \(1 \leq k \leq m, k \in \N\), let \(\delta > 0\). \(\{\{x_k\}\}\) is a \(\delta\)-cover for \(\{x_k\}\) since \(x_k \in \{x_k\}\) and \(|\{x_k\}| = |x_k - x_k| = 0 < \delta\). Therefore,
    \begin{align*}
        \h_\delta^0(\{x_k\}) &= \inf \left\{\sum_{i = 1}^{+\infty} |U_i|^0 : \{U_i\} \text{ is a } \delta\text{-cover of }\{x_k\}\right\}\\
        &\leq \sum_{i = 1}^{1} 0^0\\
        &= 1.
    \end{align*}

    Notice that any cover of \(x_k\) must contain a set, since \(x_k \notin \emptyset\). This means \(\sum_{i = 1}^{+\infty} |U_i|^0 \geq 1\). So \(\h_\delta^0(\{x_k\}) = 1\) for all \(\delta > 0\), which then implies \(\h^0(\{x_k\}) = \lim_{\delta \to 0} \h_\delta^0(\{x_k\}) = 1\).

    By theorem 2.1.1, we must have
    \begin{align*}
        \h^0(E) &= \h^0(\{x_1\}) + \h^0(\{x_2\}) + \ldots + \h^0(\{x_m\})\\
        &= 1 + 1 + \ldots + 1\\
        &= m\\
        &= \# E,
    \end{align*}
    so if \(E\) is finite, \(\h^0(E)\) is simply the counting measure.

    \item \textbf{\(E\) is infinite.} As shown in the first case, for any point \(x \in \R^n\), we have \(\h^0(\{x\}) = 1\). B.W.O.C. suppose that \(\h^0(E)\) is finite, and say it is equal to some \(a \in \R\).

    We choose \(x_1, x_2, \ldots, x_{\floor{a} + 1} \in E\) which are pairwise distinct. This is always possible since \(E\) is infinite. Let \(A = \{x_1, x_2, \ldots, x_{\floor{a}+1}\} \subset E\). By the first case, \(\h^0(E) = \# A = \floor{a} + 1 > a = \h^0(E)\). But by theorem 2.1.1 and problem 2.0.3, \(A \subset E \implies \h^0(A) \leq \h^0(E)\) which gives a contradiction. So \(\h^0(E)\) cannot be finite and must be infinity. And also \(\# E = \infty\) by the definition of the counting measure.
\end{itemize}

Therefore, \(\h^0\) is simply the counting measure.
\newpage

\section{Problem 4.0.1}
\textit{Proof:} \begin{enumerate}
    \item Let \(k = \ceil{\log_c(\epsilon/|x-y|)} + 1\) is finite. Notice that 
    \begin{align*}
        |f^{(k)}(x) - f^{(k)}(y)| &= c|f^{(k-1)}(x) - f^{(k-1)}(y)|\\
        &= c^2|f^{(k-2)}(x) - f^{(k-2)}(y)|\\
        &= \cdots\\
        &= c^{k-1}|f(x) - f(y)|\\
        &= c^k |x - y|\\
        &\leq c^{\log_c(\epsilon/|x-y|)+1} |x-y|\\
        &= c\epsilon\\
        &< \epsilon.
    \end{align*}
    
    \item Let \(f: \R \to \R\) s.t.
    \[
        f(x) = \frac{x-\sqrt{x^2+4}}{2}.
    \]
    (This is a branch of the hyperbola with \(y=x\) and \(y=0\) as asymptotes, the branch below both of them.)

    Notice that \(f\) is increasing and concave. This is because
    \begin{align*}
        f'(x) &= \frac{1}{2} - \frac{2x}{4\sqrt{x^2+4}}\\
        &= \frac{1}{2} - \frac{x}{2\sqrt{x^2+4}}\\
        &= \frac{\sqrt{x^2+4}-x}{2\sqrt{x^2+4}}\\
        &> \frac{\sqrt{x^2}-x}{2\sqrt{x^2+4}}\\
        &= \frac{|x|-x}{2\sqrt{x^2+4}}\\
        &\geq 0
    \end{align*}
    and so \(f'(x)>0\), and
    \begin{align*}
        f''(x) &= -\frac{1}{2} \cdot \frac{\sqrt{x^2+4} \cdot 1 - x \cdot \frac{1}{2} \cdot 2x \cdot \frac{1}{\sqrt{x^2+4}}}{x^2+4}\\
        &= -\frac{\sqrt{x^2+4} - \frac{x^2}{\sqrt{x^2+4}}}{2(x^2+4)}\\
        &= -\frac{(x^2+4) - x^2}{2(x^2+4)\sqrt{x^2+4}}\\
        &= -\frac{2}{(x^2+4)\sqrt{x^2+4}}\\
        &< 0,
    \end{align*}
    and so \(f''(x) < 0\).
    
    Also, notice that
    \begin{align*}
        f'(x) - 1 &= \frac{\sqrt{x^2 + 4} -x }{2\sqrt{x^2 + 4}} - 1\\
        &= - \frac{\sqrt{x^2 + 4} +x }{2\sqrt{x^2 + 4}}\\
        &<0,
    \end{align*}
    so we can show that \(f'(x) < 1\).
    
    W.L.O.G. let \(x > y\), since \(f''(y) < 0\), we will know that
    \[
        f(x) - f(y) < (x - y)f'(y) < (x-y)
    \]
    and so \(|f(x) - f(y)| < |x-y|\).
    
    Notice at the same time,
    \[
    f(x) - x = -\frac{x+\sqrt{x^2+4}}{2} < 0
    \]
    so there does not exist \(x\) such that \(f(x) = x\).
\end{enumerate}
\newpage

\section{Problem 4.0.2}
\textit{Proof:} To show that $f(K)$ is compact given $f$ is a contraction and $K$ is compact, we show that $f(K)$ is closed and bounded.

\begin{itemize}
\item $f(K)$ is bounded: since $K$ is bounded, there exists some $r > 0$ s.t. $K \subset B_{(0, \ldots, 0)} (r)$. In other words, for all $x \in K$, $|x| < r$ .
    
    Since $f$ is a contraction, set $y = 0$, we will see that $|f(x)| = c|x| < cr$. Therefore, $f(K) \subset B_{(0, \ldots, 0)} (cr)$, which shows $f(K)$ is bounded.
\item $f(K)$ is closed: since $K$ is closed, $\bar{K}$ is open.
\end{itemize}
\newpage

\section{Problem 4.0.3}
\textit{Proof:} Please add the solution to solutions/section-4-0/q-4-0-3.tex.
\newpage

\section{Problem 4.0.4}
\textit{Proof:} For this question, due to the constructive proof of the existence of a attractor for any IFS, we just have to show that the IFS's union process is equivalent to the iterative process of the iterative definition of the fractal, and that the starting point is compact.

\begin{itemize}
    \item \textbf{Sierpinski Carpet.} An equivalent definition of the Sierpinski Carpet is by starting with \([0, 1] \times [0, 1]\), and each time shrinking the size of the whole shape with scale factor \(\frac{1}{3}\) and creating 8 copies of that. Therefore, we can define the IFS \((f_1, \ldots, f_8)\) on \(\R^2 \to \R^2\):
    \begin{align*}
        f_1((x, y)) &= \left(\frac{1}{3}x, \frac{1}{3}y\right),\\
        f_2((x, y)) &= \left(\frac{1}{3}x + \frac{1}{3}, \frac{1}{3}y\right),\\
        f_3((x, y)) &= \left(\frac{1}{3}x + \frac{2}{3}, \frac{1}{3}y\right),\\
        f_4((x, y)) &= \left(\frac{1}{3}x, \frac{1}{3}y + \frac{1}{3}\right),\\
        f_5((x, y)) &= \left(\frac{1}{3}x + \frac{2}{3}, \frac{1}{3}y + \frac{1}{3}\right),\\
        f_6((x, y)) &= \left(\frac{1}{3}x, \frac{1}{3}y + \frac{2}{3}\right),\\
        f_7((x, y)) &= \left(\frac{1}{3}x + \frac{1}{3}, \frac{1}{3}y  + \frac{2}{3}\right),\\
        f_8((x, y)) &= \left(\frac{1}{3}x + \frac{2}{3}, \frac{1}{3}y  + \frac{2}{3}\right).
    \end{align*}

    Here, \(f_1\) to \(f_8\) generates the bottom-left, bottom, bottom-right, left, right, top-left, top, top-right \(\frac{1}{3}\) by \(\frac{1}{3}\) respectively. We can verify this simply by plugging two values for the first iteration (say, \((0, 0)\) and \((1, 0)\)) since they are affine. This is an IFS since \(f_1\) to \(f_8\) are all shrinking, in fact with contraction ratio \(\frac{1}{3}\)
    
    It is not difficult to see that \(f(X) = \bigcup_{i = 1}^{8} f_i(X)\) maps \(\square_k\) to \(\square_{k + 1}\) for \(k \in \N\) since this is exactly the definition. Therefore, the attractor of this IFS is
    \[
        \square = \bigcap_{i \in \N} f^{(n)}(\square_0)
    \]
    since \(\square_0 = [0, 1]^2\) is compact. This is exactly the Sierpinski Carpet \(\square\).

    \item \textbf{Koch Curve.} We can see that the iterative process of the Koch curve can be split into four pieces: the two pieces that got shrunken with scale factor \(\frac{1}{3}\) and gets copied four times, two of which are then rotated into the correct position. Therefore, define the IFS \((f_1, f_2, f_3, f_4)\) on \(\R^2 \to \R^2\):
    \begin{align*}
        f_1((x, y)) &= \left(\frac{1}{3}x, \frac{1}{3}y\right),\\
        f_2((x, y)) &= \left(\frac{1}{6}x - \frac{\sqrt{3}}{6}y + \frac{1}{3}, \frac{\sqrt{3}}{6}x + \frac{1}{6}y\right),\\
        f_3((x, y)) &= \left(\frac{1}{6}x + \frac{\sqrt{3}}{6}y + \frac{1}{2}, -\frac{\sqrt{3}}{6}x + \frac{1}{6}y + \frac{\sqrt{3}}{6}\right),\\
        f_4((x, y)) &= \left(\frac{1}{3}x + \frac{2}{3}, \frac{1}{3}y\right).
    \end{align*}

    Here, \(f_1\) generates the left part, \(f_4\) generates the right part, \(f_2\) generates the part tilted counter clockwise by \(\frac{\pi}{3}\), and \(f_3\) generates the part tilted clockwise by \(\frac{\pi}{3}\). This is an IFS since \(f_1\) to \(f_4\) are all shrinking, in fact with contraction ratio \(\frac{1}{3}\).

    It is not difficult to see that \(f(X) = \bigcup_{i = 1}^{4} f_i(X)\) maps \(K_k\) to \(K_{k + 1}\) for \(k \in \N\) since this is exactly the definition. Therefore, the attractor of this IFS is
    \[
        K = \bigcap_{i \in \N} f^{(n)}(K_0)
    \]
    since \(K_0 = [0, 1]\) is compact. This is exactly the Koch Curve \(K\).

    \item \textbf{Minkowski Sausage.} We can see that the iterative process of the Minkowski Sausage can be split into 8 pieces, all shrunken with scale factor \(\frac{1}{4}\) and affine transformed into the correct position. Therefore, define the IFS \((f_1, \ldots, f_8)\) on \(\R^2 \to \R^2\):
    \begin{align*}
        f_1((x, y)) &= \left(\frac{1}{4}x, \frac{1}{4}y\right),\\
        f_2((x, y)) &= \left(-\frac{1}{4}y + \frac{1}{4}, \frac{1}{4}x\right),\\
        f_3((x, y)) &= \left(\frac{1}{4}x + \frac{1}{4}, \frac{1}{4}y + \frac{1}{4}\right),\\
        f_4((x, y)) &= \left(\frac{1}{4}y + \frac{1}{2}, -\frac{1}{4}x + \frac{1}{4}\right),\\
        f_5((x, y)) &= \left(\frac{1}{4}y + \frac{1}{2}, -\frac{1}{4}x - \frac{1}{4}\right),\\
        f_6((x, y)) &= \left(\frac{1}{4}x + \frac{1}{2}, \frac{1}{4}y - \frac{1}{4}\right),\\
        f_7((x, y)) &= \left(-\frac{1}{4}y + \frac{3}{4}, \frac{1}{4}x - \frac{1}{4}\right),\\
        f_8((x, y)) &= \left(\frac{1}{4}x + \frac{3}{4}, \frac{1}{4}y\right).\\
    \end{align*}

    Here, \(f_1, f_3, f_6, f_8\) generates the four horizontal parts, \(f_2, f_7\) generates the two counter-clockwise rotations, and \(f_4, f_5\) generates the two clockwise rotations. This is an IFS since \(f_1\) to \(f_8\) are all shrinking, in fact with contraction ratio \(\frac{1}{4}\).

    It is not difficult to see that \(f(X) = \bigcup_{i = 1}^{8} f_i(X)\) maps \(M_k\) to \(M_{k + 1}\) for \(k \in \N\) since this is exactly the definition. Therefore, the attractor of this IFS is
    \[
        M = \bigcap_{i \in \N} f^{(n)}(M_0)
    \]
    since \(M_0 = [0, 1]\) is compact. This is exactly the Minkowski Triangle \(M\).
\end{itemize}
\newpage

\section{Problem 4.1.1}
\textit{Proof:} Consider the open set \(U = (0,1) \times \left(0, \frac{\sqrt{3}}{2}\right).\)

We have
\begin{align*}
    f_1(U) &= \left(0, \frac{1}{2}\right) \times \left(0, \frac{\sqrt{3}}{4}\right),\\
    f_2(U) &= \left(\frac{1}{2}, 1\right) \times \left(0, \frac{\sqrt{3}}{4}\right),\\
    f_3(U) &= \left(\frac{1}{4}, \frac{3}{4}\right) \times \left(\frac{\sqrt{3}}{4}, \frac{\sqrt{3}}{2}\right).
\end{align*}

It is not difficult to see that \(f_1(U) \cap f_2(U) = f_1(U) \cap f_3(U) = f_2(U) \cap f_3(U) = \emptyset\) due to obviously separated \(x\) or \(y\) intervals, and they are all contained within \(U\). So this \(U\) is an open set, and this IFS satisfies the open set condition.

Notice that \(f_1, f_2, f_3\) all have contraction ratio \(c_1 = c_2 = c_3 = \frac{1}{2}\) as shown in the definition. Solving
\[
\sum_{i = 1}^{3} \frac{1}{2^s} = 1 \iff 3 = 2^s \iff s = \frac{\log 3}{\log 2},
\]
which means the Hausdorff dimension of this IFS's attractor is \(\frac{\log 3}{\log 2}\).
\newpage

\section{Problem 4.1.2}
\textit{Proof:} If we recall the IFS of the Cantor Set is \((f_1, f_2)\) with
\begin{align*}
    f_1(x) &= \frac{1}{3}x,\\
    f_2(x) &= \frac{1}{3}x + \frac{2}{3},
\end{align*}
we notice this satisfies the OSC by taking \(U = (0, 1)\) (the open interval, which is an open set in \(\R\)), and \(f_1(U) = \left(0, \frac{1}{3}\right), f_2(U) = \left(\frac{2}{3}, 1\right)\). The contraction ratios are \(c_1 = c_2 = \frac{1}{3}\) from the equations.

Therefore, solving
\[
\sum_{i = 1}^{2} \frac{1}{3^s} = 1 \iff 2 = 3^s \iff s = \frac{\log 2}{\log 3},
\]
which means the Hausdorff dimension of the Cantor Set is \(\frac{\log 2}{\log 3}\) as desired.

If we recall the IFS of the Koch Curve is \((f_1, f_2, f_3, f_4)\) with
\begin{align*}
    f_1((x, y)) &= \left(\frac{1}{3}x, \frac{1}{3}y\right),\\
    f_2((x, y)) &= \left(\frac{1}{6}x - \frac{\sqrt{3}}{6}y + \frac{1}{3}, \frac{\sqrt{3}}{6}x + \frac{1}{6}y\right),\\
    f_3((x, y)) &= \left(\frac{1}{6}x + \frac{\sqrt{3}}{6}y + \frac{1}{2}, -\frac{\sqrt{3}}{6}x + \frac{1}{6}y + \frac{\sqrt{3}}{6}\right),\\
    f_4((x, y)) &= \left(\frac{1}{3}x + \frac{2}{3}, \frac{1}{3}y\right).
\end{align*}
we notice that this satisfies the OSC by taking \(U\) to be the interior of the triangle with vertices at \((0, 0)\), \((1, 0)\) and \(\left(\frac{1}{2}, \frac{\sqrt{3}}{6}\right)\). We denote the interior triangle formed by vertices \(A\), \(B\) and \(C\) to be \(\Delta(A, B, C)\). Therefore, \(U = \Delta\left((0, 0), (1, 0), \left(\frac{1}{2}, \frac{\sqrt{3}}{6}\right)\right)\),
\begin{align*}
    f_1(U) &= \Delta\left((0, 0), \left(\frac{1}{3},0\right), \left(\frac{1}{6}, \frac{\sqrt{3}}{18}\right)\right),\\
    f_2(U) &= \Delta\left(\left(\frac{1}{3},0\right), \left(\frac{1}{2}, \frac{\sqrt{3}}{6}\right), \left(\frac{1}{3}, \frac{\sqrt{3}}{9}\right) \right),\\
    f_3(U) &= \Delta\left(\left(\frac{1}{2}, \frac{\sqrt{3}}{6}\right), \left(\frac{2}{3},0\right), \left(\frac{2}{3}, \frac{\sqrt{3}}{9}\right)\right),\\
    f_4(U) &= \Delta\left(\left(\frac{2}{3}, 0\right), (1,0), \left(\frac{5}{6}, \frac{\sqrt{3}}{18}\right)\right),\\
\end{align*}
which looks like the following on a diagram. It can be clearly seen that they do not intersect each other, and they are all subsets of \(U\). So this IFS satisfies the OFC.

\begin{center}
    \includegraphics[scale=0.5]{solutions/section-4-1/diag-4-1-2.png}
\end{center}

The contraction ratios are \(c_1 = c_2 = c_3 = c_4 = \frac{1}{3}\) from the definition of a Koch curve.

Therefore, solving
\[
\sum_{i = 1}^{4} \frac{1}{3^s} = 1 \iff 4 = 3^s \iff s = \frac{\log 4}{\log 3},
\]
which means the Hausdorff dimension of the Koch curve is \(\frac{\log 4}{\log 3}\).
\newpage

\section{Problem 4.1.3}
\textit{Proof:} B.W.O.C. assume the Cantor Set \(C\) was countable. By definition of a countable set, we could list the points as \(\{a_i\}_{i=1}^{+\infty}\). Recall that for \(s,\delta > 0\), 

\[\h_\delta^s (C) = \inf \left\{\sum_i |U_i|^s : \{U_i\}_i \text{ is a } \delta\text{-cover of } C \right\}.\]

Consider the cover \(\{A_i\}_i(\epsilon)\) for \(\epsilon > 0\), where \(A_i\) is the open interval of size \(\frac{\epsilon}{2^i}\) centred at \(a_i\), i.e., \(A_i = B_{a_i} \left(\frac{\epsilon}{2^i}\right)\). Since each point has an open interval covering it, this is clearly a cover of \(a_i\); and we can make the first set have diameter \(|A_1| < \delta\) by taking \(\epsilon\) to be arbitrarily small, so the it satisfies the condition, and since the set sizes monotonically decrease the rest of them also satisfy the condition. Thus the cover is a \(\delta\)-cover, and therefore
\begin{align*} 
 \h_\delta^s (C) &\leq \sum_{i=0}^\infty \left(\frac{\epsilon}{2^n}\right)^s \\
 &= \epsilon^s \sum_{i=0}^{\infty} (2^{-s})^n \\
 &= \frac{\epsilon^s}{1-2^{-s}}.
\end{align*}
Since \(s > 0\), this converges, and taking \(\epsilon\) arbitrarily small we can make this arbitrarily close to 0. This is true for each \(\delta\), so \(\h^s(C) \leq 0\). By 2.1.4.1, \(\h^s(C) \geq 0\), so \(\h^s(C) = 0\). Hence the Hausdorff dimension of \(C\) is \(\leq s\) for all \(s > 0\). But taking \(s \ll \frac{\log(2)}{\log(3)}\), we know that the Hausdorff dimension of \(C\) \( \ll \frac{\log(2)}{\log(3)}\). However by Theorem 4.1.2 \(\frac{\log(2)}{\log(3)}\) was the unique Hausdorff dimension so this leads to a contradiction, and \(C\) cannot be countable.
\newpage

\section{Problem 4.1.4}
\textit{Proof:} We aim to construct an IFS in \(\R^k\) where \(k = \ceil{\frac{m}{n}}\), and will be concerning the open set condition with the open unit cube \((0, 1)^k \subset \R^k\).

We let the contraction ratio be \(c = 2^{-n}\), and we let the number of contractions be \(p = 2^m\). Now, if there exists an IFS such that the OSC is satisfied, we will have the dimension \(s\) satisfies that
\[
p \cdot c^s = 1 \iff 2^m \cdot \left(2^{-n}\right)^s = 1 \iff 2^{m-ns} = 1 \iff s = \frac{m}{n}.
\]

Now the task remaining is to find some IFS that satisfies the OSC. But this is straightforward: for each contraction in the IFS, we first apply \(x \mapsto 2^{-n}x\) to the input, which means the unit cube \((0, 1)^k\) will be mapped to \(\left(0, 2^{-n}\right)^k\).

It is well-known that one could place \((2^n)^k = 2^{nk}\) cubes of size congruent to \(\left(0, 2^{-n}\right)^k\) in the unit cube \((0, 1)^k\) without overlapping, simply by splitting the unit cube into small cubes with side length \(2^{-n}\). Each side will be divided into \(2^n\) pieces, which gives the result number of cubes \(2^{nk}\) as we desired. We only have \(2^m\) contractions, and since we have \(k = \ceil{\frac{m}{n}} \geq \frac{m}{n}\), this means \(m \leq nk\) and therefore the placement of \(2^m\) contractions is possible in this case, such that the image \((0, 1)^k\) of any two contractions do not overlap, and is still a subset of \((0, 1)^k\).

Therefore for some IFS, \((0, 1)^k\) satisfies the OSC with the desired number of contractions and contraction ratio. meaning the dimension of the IFS can take any rational number. And since every IFS has an attractor \(E \subset \R^n\), we know that there must exist some set \(E \subset \R^n\) that has a Hausdorff dimension of \(\frac{m}{n}\).

But since postive rational numbers \(\Q_+\) is dense in positive real numbers \(\R_+\), we can now conclude that the Hausdorff dimension can be arbitrarily close to any positive real number \(r > 0, r \in \R\).
\newpage

\section{Problem 4.1.5}
\textit{Proof:} Consider the IFS \((f_1, f_2)\) defined on \(\R \to \R\) as follows:
\begin{itemize}
    \item \(f_1(x) = \frac{2}{3}x\),
    \item \(f_2(x) = \frac{2}{3}x + \frac{1}{3}\).
\end{itemize}

Note that they have contraction ratios \(c_1 = c_2 = \frac{2}{3}\).

Notice that \(D = [0, 1]\) is an attractor of this IFS, since
\begin{align*}
f_1([0, 1]) &= \left[0, \frac{2}{3}\right],\\
f_2([0, 1]) &= \left[\frac{1}{3}, 1\right],
\end{align*}
and therefore \(f_1(D) \cup f_2(D) = [0, 1] = D\).

The Hausdorff dimension of \([0, 1]\) is \(s = 1\). This can be shown by considering the IFS \((g_1, g_2)\) on \(\R\):
\begin{itemize}
    \item \(g_1(x) = \frac{1}{2}x\),
    \item \(g_2(x) = \frac{1}{2} x + \frac{1}{2}\).
\end{itemize}

Both \(g_1\) and \(g_2\) have contraction ratios \(\frac{1}{2}\), and its is easy to see it satisfies the OSC by taking \((0, 1)\) as the open set, and we can see as well that \([0, 1]\) is the attractor. Therefore, the Hausdorff dimension of \([0, 1]\) \(s'\) must satisfy that \(2 \cdot \frac{1}{2}^s = 1\), which gives \(s = 1\).

However, if the OSC holds, then the dimension should also satisfy that
\[
\sum_{i = 1}^{2} \left(\frac{2}{3}\right)^s = 1,
\]
which is clearly not the case. Therefore, the OSC does not hold for this IFS.
\newpage

\section{Problem 4.2.1}
\textit{Proof:} \begin{itemize}
    \item \textbf{The Hausdorff dimension of \(\R\) is 1.} The Hausdorff dimension of \([0, 1]\) is \(s = 1\). This can be shown by considering the IFS \((f_1, f_2)\) on \(\R\):
    \begin{itemize}
        \item \(f_1(x) = \frac{1}{2}x\),
        \item \(f_2(x) = \frac{1}{2} x + \frac{1}{2}\).
    \end{itemize}
    
    Both \(f_1\) and \(f_2\) have contraction ratios \(\frac{1}{2}\), and its is easy to see it satisfies the OSC by taking \((0, 1)\) as the open set, and we can see as well that \([0, 1]\) is the attractor. Therefore, the Hausdorff dimension of \([0, 1]\) \(s'\) must satisfy that \(2 \cdot \frac{1}{2}^{s'} = 1\), which gives \(s' = 1\).
    
    By problem 2.1.8, we know that \(\R \subset \R^1\) so the Hausdorff dimention of \(\R\) is \(\leq 1\).
    
    By theorem 2.1.1, the Hausdorff measure is a measure on \(\R^n\) over the Borel \(\sigma\)-algebra, which contains \([0, 1]\) and \(\R\), and so since \([0, 1] \subset \R\), we know that for all \(s > 0\), we know that
    \[
    \h^s([0, 1]) \leq \h^s(\R).
    \]
    
    From problem 2.1.7, we know
    \[
    \text{Hausdorff dimension of } E = \inf_{s \geq 0}{\h^s(E) = 0} = \sup_{s \geq 0}{\h^s(E) = +\infty}.
    \]
    
    The Hausdorff dimension of \([0, 1]\) being \(1\) implies that for all \(s < 1\),
    \[
    +\infty = \h^s([0, 1]) \leq \h^s(\R),
    \]
    and therefore 
    \[
    \text{Hausdorff dimension of } \R = \sup_{s \geq 0}{\h^s(\R) = +\infty} \geq 1.
    \]
    
    Since the Hausdorff dimension of \(\R\) is \(\geq 1\) and \(\leq 1\), we conclude that the Hausdorff dimension of \(\R = 1\).

    \item \textbf{The 1-dimensional Hausdorff measure of \(\R\) is \(+\infty\).} We first show that
    \[\h^1 ([0, 1]) = 1.\]
    
    \begin{itemize}
        \item \textbf{Upper bound: \(\h^1([0, 1]) \leq 1\).} To show this, we just have to show that for every \(\delta\), we have some \(\delta\)-cover \(\{U_i\}\) that satisfies \(\sum_{i}|\{U_i\}|=1\). We just consider the case where \(0 < \delta < 1\). Consider the following construction for \(\{U_i\}\) where \(0 < \epsilon < \delta\):
        \[
        \{[0, (\delta - \epsilon)], [(\delta - \epsilon), 2(\delta - \epsilon)], \ldots, [(n-1) (\delta - \epsilon), n(\delta - \epsilon)], [n(\delta - \epsilon), 1]\},
        \]
        where \(n = \floor{\frac{1}{\delta - \epsilon}}\). It is not difficult to see for this construction,
        \begin{align*}
            &\phantom{=}\sum_{i} |U_i|\\
            &= [(\delta - \epsilon) - 0] + \ldots + [n(\delta - \epsilon) - (n-1) (\delta - \epsilon)] + [1 - n(\delta - \epsilon)]\\
            &= 1.
        \end{align*}
        which by definition implies that \(\h_\delta^1([0, 1]) \leq 1\) for any \(0 < \delta < 1\), and hence \(\h^1([0, 1]) \leq 1\).
        \item \textbf{Lower bound: \(\h^1([0, 1]) \geq 1\).} We consider the following definition of a measure (
    \end{itemize}
    
    By problem 2.1.6 on the translation invariance property of the \(s\)-dimensional Hausdorff measure, we know that \(\h^1([n, n + 1]) = 1\) for all \(n \in \Z\).
    
    Since
    \[\R \supset \bigcup_{n = -\infty}^{+\infty} [2n, 2n + 1]\]
    where the right-hand-side is a disjoint union and since \(s\)-dimensional Hausdorff measure is a measure on the Borel \(\sigma\)-algebra by theorem 2.1.1, it will satisfy Axiom 3 for a measure (for all the sets concerning below, since they are all Borel sets), and hence
    \begin{align*}
        \h^1(\R) &\geq \sum_{n = -\infty}^{+\infty} \h^1([2n, 2n + 1])\\
        &= \sum_{n = -\infty}^{+\infty} 1\\
        &= +\infty
    \end{align*}
    which then implies \(\h^1(\R) = +\infty\)which finishes our proof.
\end{itemize}



\newpage

\section{Problem 4.2.2}
\textit{Proof:} We alter slightly the definition of the natural measure. Specifically, we assign an uneven amount of measure on the left interval created and the right interval created each step. Let \(0 < p < 1\) such that \(p \cdot 3^s > 1\) where \(s = \log 2 / \log 3\), i.e., \(p > \frac{1}{2}\). Let the IFS of the Cantor set be \((f_1, f_2)\) where \(f_1(x) = \frac{1}{3}x\) and \(f_2(x) = \frac{1}{3} x + \frac{2}{3}\), and we denote \(f_{i_1, i_2, \ldots, i_m} (E)\) to be \(f_{i_1}(f_{i_2}(\ldots(f_{i_m}(E))))\), where each \(i_k \in \{1, 2\}\). Let \(t_1 = p, t_2 = 1 - p\), and define the measure \(\mu\) such that
\[
\mu\left(f_{i_1, i_2, \ldots, i_m} (C)\right) = \mu\left(f_{i_1, i_2, \ldots, i_m} ([0, 1]) \cap C\right) = \prod_{j = 1}^{m} t_{i_j} = t_{i_1} t_{i_2} \ldots t_{i_m}.
\]

This means every time a left interval is created, it takes up \(p\) of the measure of the original interval, with the right interval taking up \(1 - p\) of the measure.

Now, we can generalise this for any \(U \subset C\) simply using the similar concept as before, by approximating it with smaller and smaller intervals \(\cap C\):
\[
\mu(U) = \inf\left\{\sum_{i = 1}^{\mu(E_i)} : U \subset \bigcup_{i} E_i, E_i \text{ is an interval in the construction}\right\}.
\]

It is not difficult to see the non-negativity and the measure of the empty set being zero from the definition. The countable additivity is also true: consider disjoint Borel subsets \(\{E_i\}\) of \(C\), we will have \(\mu (\bigcup E_i) = \sum \mu(E_i)\) since the intervals are either disjoint or contains one-another (and we obviously don't want them to contain one-another to get closer to the \(\inf\) since simply removing the smaller one won't change the validity of the cover), and so they are disjoint, and so this holds.

Now consider \(\frac{\mu(U)}{|U|^s}\) for \(U = f_{1, 1, \ldots, 1}(C)\) where there are \(n\) 1s. It's not difficult to see that
\[
|U| = \left(\frac{1}{3}\right)^n,
\]
and that
\[
\mu(U) = p^n.
\]

Notice that
\[
\frac{\mu(U)}{|U|^s} = \frac{p^n}{3^{-sn}} = (2p)^n.
\]

Notice that this value decreases as \(p\) increases. For any \(\epsilon > 0\), we can find \(n\) that is sufficiently big enough such that \(|U| = 3^{-n} > \epsilon\), and for any \(c > 0\), we can further increase \(n\) to be big enough such that \(\mu(U) / |U|^s = (2p)^n > c\), since \(2p > 1\) and \((2p)^n\) is increasing with respect to \(n\). This means that for all \(c, \epsilon > 0\), we can find some \(U \subset C, |U| \leq \epsilon\), such that \(\mu (U) > c|U|^s\) for this measure \(\mu\) that we just constructed, as desired.
\newpage

\section{Problem 4.2.3}
\textit{Proof:} First of all, we need to determine the Hausdorff Dimension of the Cantor Dust D. We will do this by finding an IFS where it is the attractor, proving OSC, then using 4.1.2 to determine the Hausdorff Dimension.

An IFS for this set is 
 \((f_1, f_2, f_3, f_4)\) with
\begin{align*}
    f_1((x, y)) &= \left(\frac{1}{3}x, \frac{1}{3}y\right),\\
    f_2((x, y)) &= \left(\frac{1}{3}x + \frac{2}{3}, \frac{1}{3}y\right),\\
    f_3((x, y)) &= \left(\frac{1}{3}x, \frac{1}{3}y + \frac{2}{3}\right),\\
    f_4((x, y)) &= \left(\frac{1}{3}x + \frac{2}{3}, \frac{1}{3}y + \frac{2}{3}\right).\\
\end{align*}
These all have contraction ratio $\frac{1}{3}$

This satisfies the OSC: take the open interval $I := (0,1)^2$ as our open set, then 
\begin{align*}
    f_1(I) &= (0, \frac{1}{3})^2,\\
    f_2((x, y)) &= (\frac{2}{3}, 1)\times(0, \frac{1}{3})\\
    f_3((x, y)) &= (0, \frac{1}{3})\times(\frac{2}{3}, 1)\\
    f_4((x, y)) &= (\frac{2}{3}, 1)^2.\\
\end{align*}
These are all disjoint, so the OSC is satisfied. The Cantor Dust is an attractor of this IFS, and by 4.0.2 it is the attractor. By 4.1.2, if the Hausdorff dimension equals $s$ then $\sum_i c_i^s = 1 \implies \frac{4}{3^s} = 1 \implies s = \frac{\log(4)}{\log(3)}$. 

Now let's determine a mass distribution on D - by theorem 4.2.4 we can pick the natural measure. Since $c_i = \frac{1}{3}$ for all i, $c_i^s = \frac{1}{3^(\frac{\log(4)}{\log(3)})} = \frac{1}{4}$, so $\mu(I_k) = 4^{-k}$ (where $I_k$ is a square constructed in the kth iteration of the Cantor Dust).

Now let $U \subset D, |U| < \sqrt{2}$: there exists some $k$ s.t. $3^{-(k+1)} \leq |U| < 3^{-k}$. Then $U \subset I_k \cap D$ for $I_k$ defined earlier - since by this size restriction it can't be in multiple $I_k$. Hence by problem 2.0.3, 
$$ \mu(U) \leq \mu(I_k) = 4^{-k} = (3^s)^{-k}$$
$$ = (3^{-k})^s \leq (3|U|)^s,$$ 
so by 4.2.2 $\h^s(D) \geq \mu(D)*3^{-s} = \frac{1}{4} > 0$, as required.
\newpage

\section{Problem 4.2.4}
\textit{Proof:} Consider the set \(V = \left(-\frac{1}{4}, \frac{5}{4}\right) = \left(-\frac{3}{12}, \frac{15}{12}\right)\). For the IFS \((f_1, f_2)\) for the Cantor set \(C\), we have
\begin{align*}
    f_1(x) &= \frac{x}{3},\\
    f_2(x) &= \frac{x}{3} + \frac{2}{3}.
\end{align*}

Therefore, we have
\begin{align*}
    f_1(V) &= \left(-\frac{1}{12}, \frac{5}{12}\right),\\
    f_2(V) &= \left(\frac{9}{12}, \frac{13}{12}\right).
\end{align*}

Apparently \(f_1(V) \subset V, f_2(V) \subset V, f_1(V) \cap f_2(V) = \emptyset\). \(V \supset [0, 1] \supset C\) so not only does \(V\) satisfy the strong open set condition that \(C \cap V = C \neq \emptyset\), \(V\) also satisfies that \(V \supset C\) that it contains \(C\) completely.

For the Sierpinski Carpet, it is possible for \(V = (0, 1)^2\) to satisfy the SOSC. Recall that the IFS for the Sierpinski Carpet satisfies that

\begin{align*}
    g_1((x, y)) &= \left(\frac{x}{3}, \frac{y}{3}\right),\\
    g_2((x, y)) &= \left(\frac{x}{3}, \frac{y}{3} + \frac{1}{3}\right),\\
    g_3((x, y)) &= \left(\frac{x}{3}, \frac{y}{3} + \frac{2}{3}\right),\\
    g_4((x, y)) &= \left(\frac{x}{3} + \frac{1}{3}, \frac{y}{3}\right),\\
    g_5((x, y)) &= \left(\frac{x}{3} + \frac{1}{3}, \frac{y}{3} + \frac{2}{3}\right),\\
    g_6((x, y)) &= \left(\frac{x}{3} + \frac{2}{3}, \frac{y}{3}\right),\\
    g_7((x, y)) &= \left(\frac{x}{3} + \frac{2}{3}, \frac{y}{3} + \frac{1}{3}\right),\\
    g_8((x, y)) &= \left(\frac{x}{3} + \frac{2}{3}, \frac{y}{3} + \frac{2}{3}\right),\\
\end{align*}
and therefore
\begin{align*}
    g_1(V) &= \left(0, \frac{1}{3}\right) \times \left(0, \frac{1}{3}\right),\\
    g_2(V) &= \left(0, \frac{1}{3}\right) \times \left(\frac{1}{3}, \frac{2}{3}\right),\\
    g_3(V) &= \left(0, \frac{1}{3}\right) \times \left(\frac{2}{3}, 1\right),\\
    g_4(V) &= \left(\frac{1}{3}, \frac{2}{3}\right) \times \left(0, \frac{1}{3}\right),\\
    g_5(V) &= \left(\frac{1}{3}, \frac{2}{3}\right) \times \left(\frac{2}{3}, 1\right),\\
    g_6(V) &= \left(\frac{2}{3}, 1\right) \times \left(0, \frac{1}{3}\right),\\
    g_7(V) &= \left(\frac{2}{3}, 1\right) \times \left(\frac{1}{3}, \frac{2}{3}\right),\\
    g_8(V) &= \left(\frac{2}{3}, 1\right) \times \left(\frac{2}{3}, 1\right),\\
\end{align*}
and we can see they are pairwise disjoint and all subsets of \(V\). Furthermore, let \(g(E) = \bigcup_{i = 1}^{8} g_i(E)\), note that the point \(\left(\frac{1}{3}, \frac{1}{3}\right) \in V\), and also note that
\[
g_8((1, 1)) = (1, 1) \implies (1, 1) \in g^{(k)}([0, 1]^2),
\]
\[
g_1((1, 1)) = \left(\frac{1}{3}, \frac{1}{3}\right),
\]
and therefore
\[
\left(\frac{1}{3}, \frac{1}{3} \in g^{(k)}([0, 1]^2)\right),
\]
and therefore \(\left(\frac{1}{3}, \frac{1}{3}\right) \in \square\), the Sierpinski Carpet. But \(\left(\frac{1}{3}, \frac{1}{3}\right) \in V\) as well, so \(V \cap C \supset \left\{\left(\frac{1}{3}, \frac{1}{3}\right)\right\}\), and therefore \(V \cap C \neq \emptyset\).

However, it is impossible for such \(V \supset C\) to satisfy the SOSC. This is because if \(V \supset C\), then \((0, 1) \in V, (1, 1) \in V\). But notice
\[
g_4((0, 1)) = \left(\frac{1}{3}, \frac{1}{3}\right) = g_1((1, 1)),
\]
and therefore \(g_1(V) \cap g_4(V) \supset \left\{\left(\frac{1}{3}, \frac{1}{3}\right)\right\}\), and \(g_1(V) \cap g_4(V) \neq \emptyset\) which violates the pairwise disjoint condition of the OSC, hence the second part is impossible for the Sierpinski carpet.
\newpage

\section{Problem 4.2.5}
\textit{Proof:} Consider the following \(2^{n - 1}\) contractions each with contraction ration \(2^{-n}\) forming an IFS for some \(n \in N, n \geq 2\):
\begin{align*}
    (f_n)_1 (x) &= \frac{1}{2^n}(x),\\
    (f_n)_2 (x) &= \frac{1}{2^n}(x) + \frac{1}{2^n},\\
    (f_n)_3 (x) &= \frac{1}{2^n}(x) + \frac{2}{2^n},\\
    \vdots &= \vdots\\
    (f_n)_{2^{n-1}}(x) &= \frac{1}{2^n}(x) + \frac{2^{n-1} - 1}{2^n}.
\end{align*}

This satisfies the open set condition for \(U = (0, 1)\), since
\begin{align*}
    (f_n)_1 (U) &= \left(0, \frac{1}{2^n}\right),\\
    (f_n)_2 (U) &= \left(\frac{1}{2^n}, \frac{2}{2^n}\right),\\
    (f_n)_3 (U) &= \left(\frac{2}{2^n}, \frac{3}{2^n}\right),\\
    \vdots &= \vdots\\
    (f_n)_{2^{n-1}}(U) &= \left( \frac{2^{n-1} - 1}{2^n}, \frac{2^{n-1}}{2^n}\right),
\end{align*}
which are obviously pairwise disjoint and all subsets of \(U\).

Let its attractor be \(E_n\). We will see that its dimension \(s_n\) satisfies that
\[
2^{n - 1} \cdot \left(2^{-n}\right)^s = 1 \iff 2^{n - 1 - sn} = 1 \iff s = \frac{n - 1}{n}.
\]

Now, consider the set
\[
E = \bigcup_{i = 2}^{n} E_n \subset \R
\]

Since \(E \subset \R = \R^1\), we know the Hausdorff dimension of \(E \leq 1\).

Consider \(\h^1(E)\). By the countable sub-additivity property of a measure, we can see
\begin{align*}
\h^1(E) &= \h^1 \left(\bigcup_{i = 2}^{+\infty} E_n\right)\\
&\leq \sum_{i = 2}^{+\infty} \h^1(E_n).
\end{align*}

But since each \(E_n\) has a Hausdorff dimension of \(\frac{n - 1}{n} < 1\), by problem 2.1.7 we must have \(\h^1(E_n) = 0\).

Therefore, \(\h^1(E) = 0\), which shows the Hausdorff dimension of \(E\) must be \(\geq\) 1.

This shows that \(E\) has Hausdorff dimension 1 but have \(\h^1(E) = 0\).
\newpage

\section{Problem 4.2.6}
\textit{Proof:} Recall from 4.0.4 that the IFS for the Sierpinski carpet \((f_1, \ldots, f_8)\) on \(\R^2 \to \R^2\):
\begin{align*}
    f_1((x, y)) &= \left(\frac{1}{3}x, \frac{1}{3}y\right),\\
    f_2((x, y)) &= \left(\frac{1}{3}x + \frac{1}{3}, \frac{1}{3}y\right),\\
    f_3((x, y)) &= \left(\frac{1}{3}x + \frac{2}{3}, \frac{1}{3}y\right),\\
    f_4((x, y)) &= \left(\frac{1}{3}x, \frac{1}{3}y + \frac{1}{3}\right),\\
    f_5((x, y)) &= \left(\frac{1}{3}x + \frac{2}{3}, \frac{1}{3}y + \frac{1}{3}\right),\\
    f_6((x, y)) &= \left(\frac{1}{3}x, \frac{1}{3}y + \frac{2}{3}\right),\\
    f_7((x, y)) &= \left(\frac{1}{3}x + \frac{1}{3}, \frac{1}{3}y  + \frac{2}{3}\right),\\
    f_8((x, y)) &= \left(\frac{1}{3}x + \frac{2}{3}, \frac{1}{3}y  + \frac{2}{3}\right).
\end{align*}

Denote \(f(X) = \bigcup_{i = 1}^{8} f_i(X)\), \(\square_0 = [0, 1]^2\) and \(\square_k = f^{(k)}(\square_0)\), and the Sierpinski carpet as \(\square\).

The Sierpinski carpet's IFS satisfies the OSC: notice that the open unit square \(U = (0, 1)^2\) satisfies the OSC by
\begin{align*}
    f_1(U) &= \left(0, \frac{1}{3}\right) \times \left(0, \frac{1}{3}\right)\\
    f_2(U) &= \left(\frac{1}{3}, \frac{2}{3}\right) \times \left(0, \frac{1}{3}\right)\\
    f_3(U) &= \left(\frac{2}{3}, 1\right) \times \left(0, \frac{1}{3}\right)\\
    f_4(U) &= \left(0, \frac{1}{3}\right) \times \left(\frac{1}{3}, \frac{2}{3}\right)\\
    f_5(U) &= \left(\frac{2}{3}, 1\right) \times \left(\frac{1}{3}, \frac{2}{3}\right)\\
    f_6(U) &= \left(0, \frac{1}{3}\right) \times \left(\frac{2}{3}, 1\right)\\
    f_7(U) &= \left(\frac{1}{3}, \frac{2}{3}\right) \times \left(\frac{2}{3}, 1\right)\\
    f_8(U) &= \left(\frac{2}{3}, 1\right) \times \left(\frac{2}{3}, 1\right)\\.
\end{align*}
and notice they are all disjoint subsets of \(U\). Therefore, the dimension of \(\square\), \(s\), must satisfy that
\[
8 \cdot 3^{-s} = 1 \implies s = \frac{\log 8}{\log 3}.
\]

Let \(\square\) be the Sierpinski carpet, and denote \(S_k\) as a square generated in the \(k\)-th iteration of the process, for example, \(S_0 = (0, 1)^2\), note that \(S_k\) has side length \(3^{-k}\) and there are \(8^{k}\) such squares in the process. Now, we define the measure \(\mu\):
\[
\mu(S_k \cap \square) = 8^{-k},
\]
and for any \(U \subset \square\) a subset of the Sierpinski carpet, we can approximate it using some smaller squares:
\[
\mu(U) = \inf\left\{\sum_{i = 1}^{+\infty} \mu(E_i): U \subset \bigcup_{i = 1}^{+\infty} E_i, E_i \text{ is some square in the process}\right\}.
\]

Assume that \(|U| < \sqrt{1}\). There must therefore exist some \(k\) such that
\(3^{-(k + 1)} \leq |U| < 3^{-k}\) for some integer \(k\). Then, \(U \subset \left({S_k}_1 \cup {S_k}_2 \cup {S_k}_3 \cup {S_k}_4\right) \cap \square\) where \({S_k}_i\) is one of the squares in the \(k\)-th step of the construction, i.e., \(U\) can only be situated in at most 4 squares at the same time. Therefore,
\begin{align*}
    \mu(U) &\leq 4 \cdot 8^{-k}\\
    &= 4 \cdot \left(3^s\right)^{-k}\\
    &= 4 \cdot \left(3^{-k}\right)^s\\
    &= 4 \cdot 3^s \cdot \left(3^{-k - 1}\right)^s\\
    &\leq 4 \cdot 8 \cdot |U|^s\\
    &= 32 \cdot |U|^s.
\end{align*}

This means the hypothesis for the mass distribution principle is true for \(\epsilon = 1\) and \(c = 32\). Therefore, from mass distribution principle, we have
\[
\h^s(\square) \geq \frac{1}{32}.
\]
\newpage

\section{Problem 5.0.1}
\textit{Proof:} Let \(s = \frac{\log 3}{\log 2}\). We restate some definitions first: let \(I_n = (i_1, i_2, \ldots, i_n) \in \{1, 2, 3\}^n\) where \(i_k \in \{1, 2, 3\}\) for each \(1 \leq k \leq n\), and define \(S_{I_n}(\Delta) = S_{i_1}(S_{i_2}(\ldots(S_{i_n}(\Delta))))\) where \(\Delta\) is the closed equilateral triangle with vertices \((0, 0), (1, 0)\) and \(\left(\frac{1}{2}, \frac{\sqrt{3}}{2}\right)\).

We define for \(E \subset \R^2\) that
\[
S(E) = \bigcup_{i = 1}^{3} S_i(E).
\]

\begin{itemize}
    \item \textbf{Upper Bound: \(\h^s(\sierpinski) \leq  1\).} Recall the definition that
    \[
    \h^s_\delta(\sierpinski) = \inf_{\{U_i\} \text{ is a }\delta\text{-cover of }\sierpinski} \sum_{i = 1}^{+\infty} |U_i|^s.
    \]

    We can first show that \(|\triangle| \geq 1\) since \(|(0, 0) - (1, 0)| = 1\) and \(|\triangle| \leq 1\) since \(\triangle \subset \overline{B_{(0, 0)} (1)}\). And therefore, we can see that
    \[
    \left|S_{I_n}(\triangle)\right| = \frac{1}{2^n}
    \]
    from the definition of a contraction immediately.

    By theorem 4.0.2, we can see that
    \[
        \sierpinski = \bigcap_{k = 0}^{+\infty} S^{(k)}(\Delta)
    \]
    since \(\sierpinski\) is the attractor of the IFS.

    Consider a certain \(k \in \Z, k \geq 0\), and it is not difficult to see from definition that
    \[
        S^{(k)} (\Delta) = \bigcup_{I_n \in \{1, 2, 3\}^n} S_{I_n} (\Delta)
    \]
    from the definition of \(S^{(k)}\). (This can be shown by induction on \(k\).)

    Therefore, we can see
    \[
        \sierpinski \subset \bigcup_{I_n \in \{1, 2, 3\}^n} S_{I_n} (\Delta),
    \]
    and therefore, \(\{S_{I_n} (\Delta)\}\) gives a \(\delta\)-cover for \(\sierpinski\), when \(\delta > \frac{1}{3^n} \iff 3^n > \frac{1}{\delta} \iff n > -\log_3(\delta)\).

    Also, notice that \(\# \{S_{I_n} (\Delta)\} = 3^n\).

    Therefore, for any \(0 < \delta < 1\), choose \(n = \floor{-\log_3(\delta)} + 1 \implies n > -\log_3(\delta)\), and therefore
    \begin{align*}
        \h^s_\delta(\sierpinski) &= \inf_{\{U_i\} \text{ is a }\delta\text{-cover of }\sierpinski} \sum_{i = 1}^{+\infty} |U_i|^s\\
        &\leq \sum_{i = 1}^{3^n} \left(\frac{1}{2^n}\right)^{s}\\
        &= 3^n \cdot \frac{1}{2^{\left(n \cdot \log_2(3)\right)}}\\
        &= 3^n \cdot \frac{1}{3^n}\\
        &= 1,
    \end{align*}
    and therefore
    \[
        \h^s(\sierpinski) = \lim_{\delta \to 0} \h^s_\delta(\sierpinski) \leq 1,
    \]
    which shows the upper bound as desired.

    \item We show that the IFS of the Sierpinski triangle satisfies the OCS. Consider the open set \(U\) which is the open triangle with vertices \((0, 0), (1, 0), \left(\frac{1}{2}, \frac{\sqrt{3}}{2}\right)\). We notice that
    \begin{align*}
        S_1(U) &\text{ is an open triangle with vertices } (0, 0), \left(\frac{1}{2}, 0\right), \left(\frac{1}{4}, \frac{\sqrt{3}}{4}\right),\\
        S_2(U) &\text{ is an open triangle with vertices } \left(\frac{1}{2}, 0\right), \left(1, 0\right), \left(\frac{3}{4}, \frac{\sqrt{3}}{4}\right),\\
        S_3(U) &\text{ is an open triangle with vertices } \left(\frac{1}{4}, \frac{\sqrt{3}}{4}\right), \left(\frac{3}{4}, \frac{\sqrt{3}}{4}\right), \left(\frac{1}{2}, \frac{\sqrt{3}}{2}\right),
    \end{align*}
    and therefore they are pairwise disjoint and are all subsets of \(U\). Therefore the IFS satisfies the open set condition.
    
    Now, we show that the natural measure on \(\sierpinski\) satisfies the hypothesis for the mass distribution principle, for \(\epsilon = 1\) and \(c = 6\).

    Consider some \(U \subset \sierpinski\) such that \(|U| < 1\). There must exist some \(k\) such that \(2 \cdot 2^{-(k+1)} \leq |U| < 2 \cdot 2^{-k}\) for some \(k \geq 1\). Then, \(U \subset \bigcup_{i = 1}^{6} \left(\Delta_{k}\right)_i \cap \sierpinski\), where \(\Delta_{t}\) is a triangle formed in the \(t\)'s iteration of the IFS. This means that \(U\) is in at most \(6\) triangles in the \(k\)-th iteration, as shown in the diagram below:

    \begin{center}
        \includegraphics[scale=0.4]{solutions/section-5-0/diag-5-0-1.png}
    \end{center}
    
    Therefore we will have

    \begin{align*}
        \mu(U) & \leq 2 \cdot 3^{-k}\\
        &= 6 \cdot \left(2^{\log_2^3}\right)^{-(k+1)}\\
        &= 6 \cdot \left(2^{-(k+1)}\right)^{\log_2^3}\\
        &\leq 6 \cdot |U|^s.
    \end{align*}

    Notice that \(\mu(\sierpinski) = \mu(\Delta \cap \sierpinski) = \mu(S_{I_0} (\Delta) \cap \sierpinski) = 1\) and therefore,
    \[
    \h^s(\sierpinski) \geq \frac{\mu(\sierpinski)}{6} = \frac{1}{6}
    \]
    as desired.
\end{itemize}
\newpage

\section{Problem 5.0.2}
\textit{Proof:} For \(a_1\), there are only 3 possible distinct non-empty coverings: 1 triangle, 2 triangles, and 3 triangles. 

\begin{enumerate}
    \item For 1 triangle, the diameter of the set is \(\frac{1}{2}\), the proportion covered is \(\frac{1}{3}\), so the value of \(a_1\) is \(\frac{(1/2)^s}{(1/3)} = 1\);
    \item For 2 triangles, the diameter is 1, while the proportion covered is \(\frac{2}{3}\), notably less than 1: thus this will result in a higher value for \(a_1\);
    \item For 3 triangles, the diameter is 1, the proportion covered is 1, hence our value is also 1.
\end{enumerate}

Hence \(a_1 = 1\).

For \(a_2\), we need to do a similar job -- but we can ignore coverings which have points in only one of the 3 largest triangles due to these being equivalent to colourings of \(a_1\). This is clear as if we have a covering of \(T_n\), we can scale it down by scale factor \(\frac{1}{2}\) centred at \((0,0)\) to cover one of the three largest triangles of \(T_{n+1}\). This is clearly a covering as these triangles have been scaled down in a similar fashion to the first contraction function in the IFS, thus it's the same as just adding this function to the beginning of \(S_{I_n}(\Delta)\). This will have half the diameter, and have \(\frac{1}{3}\) of the measure (as there are 3 identical largest triangles that could be filled). Thus letting \(D\) be the diameter, and \(M\) be the measure, our value for \(a_{i+1}\) is \(\frac{(D/2)^s}{M/3} = \frac{D^s}{M}\), which is the same value as our covering for \(a_i\). Thus our lemma is proven. 

Therefore, there are then 2 cases to consider: if one of the corner triangles is filled, and if one of the corner triangles is not filled. 

I claim the best answer is when we pick the 6 triangles closest to the centre empty triangle: this would give an answer of
\[\frac{0.75^{\log(2)/\log(3)}}{\frac{6}{9}} = \frac{3^{\log(3)/\log(2) - 1}}{2} = 0.95075.\]

If one of the corner triangles is filled, we may as well fill out the 3 triangles in the sub-triangle of that corner triangle (as we must have one outside this sub-triangle anyway.) If we use a triangle adjacent to this sub-triangle, we may as well also have the one symmetrically opposite (as this doesn't increase the diameter) But this only covers 5 triangles with the same diameter as last time so can't be better.

If we choose one of the other corner triangles, we may as well cover the whole triangle anyway but this would just give us a value of 1. If we don't choose the corner triangles but use all the other triangles, this gives us a diameter of \(\frac{\sqrt{13}}{4}\), and a measure of \(\frac{7}{9}\): plugging this into the formula we get a value of around 1.09 which is worse than the trivial bound of 1.

The other case is if we don't pick any of the corners. Then there are really only two cases to consider: picking 3 adjacent triangles and picking all 6 triangles. If we pick 3 adjacent triangles, this gives us a diameter of greater than \(\frac{1}{2}\) but the same number of triangles as if we just picked one of the sub-triangles with 3 triangles - thus it's strictly worse than that bound. The other case is the best bound shown earlier - hence we are done and
\[a_2 = \frac{3^{\log(3)/\log(2) - 1}}{2}.\]

The diagram below shows the choice of triagnles for \(a_2\).

\begin{center}
    \includegraphics[width=0.5\linewidth]{solutions/section-5-0/diag-5-0-2.png}
\end{center}
\newpage

\section{Problem 5.0.3}
\textit{Proof:} I will prove that \(a_i \geq a_{i+1}\) -- if I could show that for any possible value of \(a_i\), there exists a covering of \(T_{n+1}\) with the same value, then we would be done (as \(a_{i+1}\) is defined as the minimum of these values, there could be one less than these but it is at most \(a_{i}\)).

However this is clear as if we have a covering of \(T_n\), we can scale it down by scale factor \(\frac{1}{2}\) centred at \((0,0)\) to cover one of the three largest triangles of \(T_{n+1}\). This is clearly a covering as these triangles have been scaled down in a similar fashion to the first contraction function in the IFS, thus it's the same as just adding this function to the beginning of \(S_{I_n}(\Delta)\). This will have half the diameter, and have \(\frac{1}{3}\) of the measure (as there are 3 identical largest triangles that could be filled). Thus letting D be the diameter, and M be the measure, our value for \(a_{i+1}\) is \(\frac{(D/2)^s}{M/3} = \frac{D^s}{M}\), which is the same value as our covering for \(a_i\). Thus our lemma is proven.
\newpage

\section{Problem 5.0.4}
\textit{Proof:} \begin{center}
    \includegraphics[scale=0.5]{solutions/section-5-0/diag-5-0-4.jpg}
\end{center}

This is one of the coverings of \(T_4\). The diameter is clearly \(\frac{13}{16}\) - as this is the distance between two opposite points in the figure. It also contains 66 out of 81 triangles - thus its value is \(\frac{\frac{13}{16}^s}{\frac{66}{81}}\) which is approximately \(0.8831093965\), notably less than \(0.9\). Hence \(a_4 \leq 0.8831093965 < 0.9\) so by theorem 5.0.2 the Hausdorff dimension is \(\leq 0.8831093965 < 0.9\), as required.
\newpage

\section{Problem 5.0.5}
\textit{Proof:} Recall the IFS \((m_1, m_2, \ldots, m_8)\) for the Minkowski Sausage \(M\) is:

\begin{align*}
    S_1((x, y)) &= \left(\frac{1}{4}x, \frac{1}{4}y\right),\\
    S_2((x, y)) &= \left(-\frac{1}{4}y + \frac{1}{4}, \frac{1}{4}x\right),\\
    S_3((x, y)) &= \left(\frac{1}{4}x + \frac{1}{4}, \frac{1}{4}y + \frac{1}{4}\right),\\
    S_4((x, y)) &= \left(\frac{1}{4}y + \frac{1}{2}, -\frac{1}{4}x + \frac{1}{4}\right),\\
    S_5((x, y)) &= \left(\frac{1}{4}y + \frac{1}{2}, -\frac{1}{4}x - \frac{1}{4}\right),\\
    S_6((x, y)) &= \left(\frac{1}{4}x + \frac{1}{2}, \frac{1}{4}y - \frac{1}{4}\right),\\
    S_7((x, y)) &= \left(-\frac{1}{4}y + \frac{3}{4}, \frac{1}{4}x - \frac{1}{4}\right),\\
    S_8((x, y)) &= \left(\frac{1}{4}x + \frac{3}{4}, \frac{1}{4}y\right).\\
\end{align*}

Notice this satisfies the OSC by considering a open square \(O\) with vertices at
\[(0, 0), (1, 0), \left(\frac{1}{2}, \frac{1}{2}\right), \left(\frac{1}{2}, -\frac{1}{2}\right),\] as shown in the following diagram.

\begin{center}
    \includegraphics[scale=0.3]{solutions/section-5-0/diag-5-0-5-1.png}
\end{center}

Therefore, the dimension of \(M\), \(m\) must satisfy that
\[
8 \cdot \frac{1}{4^m} = 1 \iff m = \frac{3}{2} = 1.5,
\]
by theorem 4.1.2

Let \(M_0 = [0, 1]\) the closed interval. Let \(I_n = (i_1, i_2, \ldots, i_n)\) be a list with each \(i_k \in \{1, 2, \ldots, 8\}\), and set
\[
S_{I_n}(M_0) = S_{i_1}\left(S_{i_2}\left(\ldots\left(S_{i_n}(M_0)\right)\right)\right),
\]

Define the natural measure as
\[
\mu(S_{I_n}(M_0) \cap M) = \left(\frac{1}{4}\right)^{mn} = 8^{-n}.
\]

Let \(\mu\) be the natural measure on \(M\), the Minkowski sausage. Define
\[
\mathcal{M}_n = \{S_{I_n}(M_0) : I_n \in \{1, 2, \ldots, 8\}^n\}
\]
be the collection of all \(8^n\) scaling of size \(4^{-n}\) of the Minkowski Sausage, and let \(\{E_i\}_i \subset \mathcal{M}_n\) be a non-empty collection of those iterations of the scaling, define
\[
b_n = \min_{\{E_i\}_i \subset \mathcal{M}_n} \left\{\frac{|\bigcup_i E_i|^m}{\mu(\bigcup_i E_i)}\right\}.
\]

\textbf{Claim.} \(\h^m(M) \leq b_n\).

\textit{Proof.} We first show \(\h_\delta^m(M) \leq b_n\) for \(\h_1^m(M)\).

Since \(b_n\) is a minimum of some finite number of possibilities (subsets of the finite set \(\mathcal{M}_n\)), so there must be a collection of set \(U\) with \(\#U = k\) that gives the minimum of \(b_n\). Let \(\mathcal{U} = \bigcup_{E_i \in U} E_i\), the union of the elements in the collection \(U\).

In the case where \(\mathcal{U} = M\) is the Minkowski Sausage, then we see that this must be a cover of the Minkowski Sausage. In this case,
\[
b_n = \frac{|M|^m}{\mu(M)} \geq \frac{1^m}{1} = 1,
\]
since notice that the Minkowski Sausage contains the points \((0, 0)\) and \((1, 0)\) and they are distance of 1 apart, and noticing that \(S_{I_0} (M) = M\) so \(\mu(M) = 1\) naturally.

On the other hand, notice that \(\h_{1}^m(M) \leq 1^m = 1\) since the set containing the closed square with vertices at \((0, 0), (1, 0), \left(\frac{1}{2}, \frac{1}{2}\right)\) covers \(M\) and has diameter \(1\), and therefore from the definition, \(\h_{1}^m(M) \leq 1\) as desired. Therefore, \(\h_1^m(M) \leq 1 \leq b_n\) and satisfies the claim as desired.

Otherwise, let \(V = \mathcal{M}_n \setminus U\), all sets that are not contained in the collection \(U\). \(\# V = 8^n - k\). If for some \(S_{I_n} (M) = E_{m}' \in V\) that is not in \(U\), we scale down of these, and place it in \(V\) in the sense \(\mathcal{M}_{2n}\), the \(2n\)-th level of the construction of \(M\). If \(U\) is scaled down to \(U'\) and \(\mathcal{U}' = \bigcup_{E_i \in U} E_i\), we will see \(|\mathcal{U}'| = 4^{-n}|\mathcal{U}|\). Therefore, there will be \(8^n - k\) copies of these \(U'\)(s), one each for each element in \(\mathcal{M}_n\) that \(U\) does not hit. However this still does not cover \(M\) (since there are still missing pieces in each of parts of \(V\)), so we repeatedly create \(U''\), \(U'''\), etc.

Therefore, we cover \(M\) using 1 copy of \(\mathcal{U}\), \((8^n - k)\) copies of \(\mathcal{U}'\), \((8^n - k)^2\) copies of \(\mathcal{U}''\), and so on. Therefore,
\begin{align*}
    \h_1^m(M) &\leq \sum_{i = 0}^{+\infty} (8^n - k)^i |\mathcal{U}^{(i)}|^m\\
    &= \sum_{i = 0}^{+\infty} (8^n - k)^i (4^{-in} |\mathcal{U}|)^m\\
    &= |\mathcal{U}|^m \sum_{i = 0}^{+\infty} \left(\frac{8^n - k}{8^n}\right)i\\
    &= \frac{|\mathcal{U}|^s}{1 - \frac{8^n - k}{8^n}}\\
    &= \frac{|\mathcal{U}|^s}{\frac{k}{8^n}}\\
    &= \frac{|\mathcal{U}|^s}{\mu(\mathcal{U})}\\
    &= b_n,
\end{align*}
the first inequality sign arising due to definition, the second due to what we just show, the third by sum properties, the fourth by geometric series, the fifth by trivial algebra, the sixth by the definition of \(\mu(\mathcal{U})\) and that \(\mathcal{U}\) being the union of \(k\) copies of the \(n\)th iteration of the Minkowski Sausage which only share at most points which has dimension \(0\) so zero \(m\)-dimension measure, the seventh equal sign simply by definition of \(b_n\).

Now, we move on to \(\delta < 1\), and there must exist some \(t \in \N\) such that \(4^{-t} < \delta\). Therefore, instead with starting the cover \(U\) states in the previous part, we simply start with \(8^t\) copies of \(U\) scaled down by \(4^{-t}\), one at each level \(t\) iteration of \(M\). Then we simply preform the same process for each scaled down copies of \(U\), and at each stage we get \(8^n\) extra copies of \(U^{(n)}\), with a diameter scaled by \(4^{-n}\), and therefore
\begin{align*}
    \h_{\delta}^m(M) &\leq 8^t \sum_{i = 0}^{+\infty} (8^n - k)^i (4^{-t} |\mathcal{U}^{(n)}|)^m\\
    &= 8^t \cdot 4^{-t \cdot \frac{3}{2}} \sum_{i = 0}^{+\infty} (8^n - k)^i (|\mathcal{U}^{(n)}|)^s\\
    &= \sum_{i = 0}^{+\infty} (8^n - k)^i (|\mathcal{U}^{(n)}|)^s\\
    &= b_n
\end{align*}
similar to earlier. The largest size we used in this cover is \(4^{-t} < \delta\) is valid. Therefore, if we take the limit \(\delta \to \infty\), we will have
\[
\h^m(M) \leq b_n.
\]

Now consider the bounds of \(b_n\) when \(n = 2\). Consider the following diagram of the second iteration of \(M\), i.e., \(\mathcal{M}_2\).

\begin{center}
    \includegraphics[scale=0.3]{solutions/section-5-0/diag-5-0-5-2.jpg}
\end{center}

It is worth mentioning why this diagram makes sense and what are the different elements of the diagram. First, note that \(M_0 = [0, 1] \subset\) the black square, which is a closed square with vertices at \((0, 0), (1, 0), \left(\frac{1}{2}, \frac{1}{2}\right), \left(\frac{1}{2}, -\frac{1}{2}\right)\). The first iteration of the black square is the eight red squares, which are all subsets of the black square. Therefore, the Minkowski Sausage \(M \subset\) the black square, and the first iteration of the Minkowski Sausage, elements in the set \(\mathcal{M}_1\), are each contained in one of the red squares. The first iteration of the red squares (the second iteration of the black square) are the orange squares, and elements in the set \(\mathcal{M}_2\) are each contained in one of the orange squares.

Now, we investigate the 10 orange squares marked in blue, and let our \(U \subset \mathcal{M}_2\) be the set of the 10 elements of \(\mathcal{M}_2\) contained in the 10 orange squares. We know that \(|\mathcal{U}| \leq |\text{blue square}| = \frac{1}{4}\) since \(\mathcal{U} \subset \) blue square.

Furthermore, each element in \(U\) has measure \(8^{-2} = \frac{1}{64}\), and therefore since there are 10 elements in \(U\), \(\mathcal{U}\) will have measure \(\frac{10}{64} = \frac{5}{32}\). Therefore,
\begin{align*}
    \h^m(M) &\leq b_2\\
    &\leq \frac{|\mathcal{U}|^m}{\mu(\mathcal{U})}\\
    &\leq \frac{4^{-\frac{3}{2}}}{\frac{5}{32}}\\
    &= \frac{\frac{1}{8}}{\frac{5}{32}}\\
    &= \frac{4}{5}\\
    &= 0.8
\end{align*}
as desired.
\newpage

\section{Problem 5.0.6}
\textit{Proof:} We show that \(a_3\) equals \(0.910411\) -- precisely
\[\frac{7^{\frac{\log(3)}{\log(2)}}}{24}.\]

This is done using this program, which brute-forces every combination of triangles to make the cover, finds the diameter, and computes the value of \(a_3\). Then it finds the minimum out of all these values and prints it out. This is extremely naive and would not work for \(a_4\).


\begin{verbatim}
#include <cmath>
#include <vector>
#include <set>
#include <tuple>
#include <algorithm>
#include <iostream>
#include <limits>

using namespace std;
using Point = pair<double, double>;
using Triangle = vector<Point>;

double root3 = sqrt(3.0);
double s = log(3.0) / log(2.0);

Point div(const Point& pt, double x) {
    return {pt.first / x, pt.second / x};
}

Point add(const Point& pt1, const Point& pt2) {
    return {pt1.first + pt2.first, pt1.second + pt2.second};
}

Point f0(const Point& pt) {
    return div(pt, 2.0);
}

Point f1(const Point& pt) {
    return add({0.5, 0.0}, div(pt, 2.0));
}

Point f2(const Point& pt) {
    return add({0.25, root3 / 4.0}, div(pt, 2.0));
}

using Func = Point (*)(const Point&);

vector<Func> funcs = {f0, f1, f2};

Triangle f(int n, const Triangle& tri) {
    Triangle result;
    for (const auto& pt : tri) {
        result.push_back(funcs[n](pt));
    }
    return result;
}

vector<Triangle> nexttri(const vector<Triangle>& tri) {
    vector<Triangle> tri1;
    for (const auto& t : tri) {
        for (int i = 0; i < 3; ++i) {
            tri1.push_back(f(i, t));
        }
    }
    return tri1;
}

set<Point> nextpts(const set<Point>& pts) {
    set<Point> pts1;
    for (const auto& pt : pts) {
        for (int i = 0; i < 3; ++i) {
            pts1.insert(funcs[i](pt));
        }
    }
    return pts1;
}

double dist(const Point& pt1, const Point& pt2) {
    return sqrt(pow(pt1.first - pt2.first, 2)
            + pow(pt1.second - pt2.second, 2));
}

double diameter(const set<Point>& pts) {
    double diam = 0.0;
    for (const auto& pt1 : pts) {
        for (const auto& pt2 : pts) {
            diam = max(diam, dist(pt1, pt2));
        }
    }
    return diam;
}

int main() {
    vector<Triangle> tri0 = {{{{0, 0}, {1, 0}, {0.5, root3 / 2}}}};
    set<Point> pts0 = {{0, 0}, {1, 0}, {0.5, root3 / 2}};

    auto tri1 = nexttri(tri0);
    auto pts1 = nextpts(pts0);
    int N = 3;

    for (int i=0;i<N-1;++i){
        tri1 = nexttri(tri1);
        pts1 = nextpts(pts1);
    }



    double mina = numeric_limits<double>::max();

    size_t numTriangles = tri1.size();

    double dd = -1, nn = -1; // store exact values
    for (size_t i = 1; i < (1u << numTriangles); ++i) {
        
        vector<Triangle> tri;
        set<Point> pts;

        for (size_t j = 0; j < numTriangles; ++j) {
            if (i & (1u << j)) {
                tri.push_back(tri1[j]);
                for (const auto& pt : tri1[j]) {
                    pts.insert(pt);
                }
            }
        }

        double d = diameter(pts);
        double n = static_cast<double>(tri.size());
        double a = pow(d, s) / (n / pow(3.0, N));
        if (a < mina) {
            mina = a;
            dd = d;
            nn = n;
        }
    }

    cout << mina << endl;
    cout << dd << " " << nn << endl;
    return 0;
}
\end{verbatim}

\newpage

\section{Problem 5.0.7}
\textit{Proof:} Please add the solution to solutions/section-5-0/q-5-0-7.tex.
\newpage
\end{document}

