Let \(E \subset \R^n, n \in \N\). Let \(A\) be the \(n\)-dimensional unit open hypercube centred at the origin, i.e., \(A = (-0.5, 0.5)^n\).

Let \(s > n\). We prove the following claim:

\textbf{Claim.} \(\h^s(A) = 0\).

\textit{Proof.} Let \(\delta > 0.\) We can partition \(A\) into closed \(n\)-dimensional hypercubes of side length \(\frac{\delta}{2\sqrt{n}}\).

It is clear that these cubes have diameter equal to the length of the diagonal that goes from \((0, 0, \ldots, 0)\) to \((1, 1, \ldots, 1)\). And therefore, by Pythagoras theorem, the diameter is \(\sqrt{\left(\frac{\delta}{2\sqrt{n}}\right)^2 \cdot n} = \frac{\delta}{2} < \delta\).

To cover one edge of \(A\), we need \(\frac{1}{\left(\frac{\delta}{2\sqrt{n}}\right)} = \frac{2\sqrt{n}}{\delta}\) cubes, so to cover all of \(A\), we will need \(\left(\frac{2\sqrt{n}}{\delta}\right)^n = \frac{\left(2\sqrt{n}\right)^n}{\delta^n}\) cubes.

Therefore,
\begin{align*}
    &\phantom{=} \sum_{i = 1}^{\frac{\left(2\sqrt{n}\right)^n}{\delta^n}} | \text{small cube}|^s\\
    &= \frac{\left(2\sqrt{n}\right)^n}{\delta^n} \cdot \left(\frac{\delta}{2}\right)^s\\
    &= \delta^{s - n} \cdot \left(\frac{\left(2\sqrt{n}\right)^n}{2^s}\right).
\end{align*}

Since these cubes cover \(A\), and each have diameter \(< \delta\), they are a \(\delta\)-cover of \(A\), and therefore, by definition,
\[
H_\delta^s(A) \leq \delta^{s - n}\left(\frac{\left(2\sqrt{n}\right)^n}{2^s}\right).
\]

\(s - n > 0\) since \(s > n\), and therefore, 
\[
\lim_{\delta \to 0} \delta^{s - n} \left(\frac{\left(2\sqrt{n}\right)^n}{2^s}\right) = 0 \implies \h^s(A) = \lim_{delta \to 0} \h_{\delta}^s(A) \leq 0.
\]

By part 1 of problem 2.1.4, we know \(\h^s(A) \geq 0\), and therefore \(\h^s(A) = 0\). \qed

Now consider \(cA = \{cx : x \in A\}\) for \(c > 0\). By problem 2.1.5, \(\h^s(cA) = c^s \h^s(A) = 0\). \(cA = \left(-\frac{c}{2}, \frac{c}{2}\right)^n\). Let
\[
B = \bigcup_{i \in \N} iA.
\]

\textbf{Claim.} \(B = \R^n\).

\textit{Proof.} \(A \subset \R^n \implies iA \subset \R^n \implies B \subset \R^n\).

Take some \(x \in \R^n\) and let \(x = (x_1, x_2, \ldots, x_n)\). Let
\[
m = \floor{\max |x_i|} + 1 \implies |x_i| < m \text{ for all } 1 \leq i \leq n.
\]

Consider \(2mA = (-m, m)^n\). Clearly \(x = \left(x_1, \ldots, x_n\right) \in 2mA\) which in turn implies that \(x \in B = \bigcup_{i \in \N} iA\), and therefore \(\R^n \subset B\).

Therefore \(B = \R^n\). \qed

By part 3 of problem 2.1.4, we must have
\[
\h^s (\R^n) \leq \sum_{i = 1}^{+\infty} \h^s(iA) = 0,
\]
but by part 1 of problem 2.1.4, \(\h^s (\R^n) \geq 0\). This means that \(\h^s(\R^n) = 0\).

By problem 2.0.3 and theorem 2.1.1, for \(E \subset \R^n\), we must have
\[
\h^s(E) \leq \h^s (\R_n) = 0,
\]
but also by part 1 of 2.1.4, we have \(\h^s(E) \geq 0 \implies \h^s(E) = 0\).

Therefore, for all \(s > n\), we have \(\h^s(E) = 0\).

B.W.O.C. assume that the Hausdorff dimension of \(E = d > n\). Therefore, for all \(r < d\), we havve \(\h^r(E) = +\infty\) by definition 2.1.2 and part 1 of problem 2.1.7.

Since \(d > n\), we may choose \(n < s < d\). \(s > n \implies \h^s(E) = 0\), but \(s < d \implies \h^s(E) = +\infty\), which is a contradiction.

Therefore, the Hausdorff dimension of \(E \leq n\), which finishes our proof of the question.