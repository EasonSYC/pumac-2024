\begin{enumerate}
    \item \(\Sigma = \p(X)\) is a \(\sigma\)-algebra because:
    \begin{enumerate}
        \item \(\emptyset \in \Sigma = \p(X)\),
        \item \(X \setminus E_i \subset X \), so \(X \setminus E_i \in \p(X)\),
        \item \(\bigcup_{i = 1}^{+\infty}E_i \subset X\), so \(\bigcup_{i = 1}^{+\infty}E_i \in \p(X)\),
        \item \(\bigcap_{i = 1}^{+\infty}E_i \subset X\), so \(\bigcap_{i = 1}^{+\infty}E_i \in \p(X)\),
    \end{enumerate}
    In short, \(\Sigma = \p(X)\) is closed under complement, intersection and union, and therefore a \(\sigma\)-algebra.
    \item \(\mu\) is a measure because \(\mu: \Sigma \to \overline{\R}\) satisfies that:
    \begin{enumerate}
        \item \(\mu(\emptyset) = \# \emptyset = 0\),
        \item for all \(E \in \p(X), \mu(E) = \# E \geq 0\),
        \item let \(\{E_i\}_{i = 0}^{+\infty}\) be pairwise disjoint,
        \begin{align*}
            \mu\left(\bigcup_{i = 1}^{+\infty} E_i\right) &= \# \bigcup_{i = 1}^{+\infty} E_i\\
            &= \sum_{i = 1}^{+\infty} \# E_i\\
            &= \sum_{i = 1}^{+\infty} \mu(E_i),
        \end{align*}
        the second equal sign being true since all the sets are pairwise disjoint.
    \end{enumerate}
\end{enumerate}