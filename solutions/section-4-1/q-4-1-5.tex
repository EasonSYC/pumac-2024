Consider the IFS \((f_1, f_2)\) defined on \(\R \to \R\) as follows:
\begin{itemize}
    \item \(f_1(x) = \frac{2}{3}x\),
    \item \(f_2(x) = \frac{2}{3}x + \frac{1}{3}\).
\end{itemize}

Note that they have contraction ratios \(c_1 = c_2 = \frac{2}{3}\).

Notice that \(D = [0, 1]\) is an attractor of this IFS, since
\begin{align*}
f_1([0, 1]) &= \left[0, \frac{2}{3}\right],\\
f_2([0, 1]) &= \left[\frac{1}{3}, 1\right],
\end{align*}
and therefore \(f_1(D) \cup f_2(D) = [0, 1] = D\).

The Hausdorff dimension of \([0, 1]\) is \(s = 1\). This can be shown by considering the IFS \((g_1, g_2)\) on \(\R\):
\begin{itemize}
    \item \(g_1(x) = \frac{1}{2}x\),
    \item \(g_2(x) = \frac{1}{2} x + \frac{1}{2}\).
\end{itemize}

Both \(g_1\) and \(g_2\) have contraction ratios \(\frac{1}{2}\), and its is easy to see it satisfies the OSC by taking \((0, 1)\) as the open set, and we can see as well that \([0, 1]\) is the attractor. Therefore, the Hausdorff dimension of \([0, 1]\) \(s'\) must satisfy that \(2 \cdot \frac{1}{2}^s = 1\), which gives \(s = 1\).

However, if the OSC holds, then the dimension should also satisfy that
\[
\sum_{i = 1}^{2} \left(\frac{2}{3}\right)^s = 1,
\]
which is clearly not the case. Therefore, the OSC does not hold for this IFS.