Recall from 4.0.4 that the IFS for the Sierpinski carpet \((f_1, \ldots, f_8)\) on \(\R^2 \to \R^2\):
\begin{align*}
    f_1((x, y)) &= \left(\frac{1}{3}x, \frac{1}{3}y\right),\\
    f_2((x, y)) &= \left(\frac{1}{3}x + \frac{1}{3}, \frac{1}{3}y\right),\\
    f_3((x, y)) &= \left(\frac{1}{3}x + \frac{2}{3}, \frac{1}{3}y\right),\\
    f_4((x, y)) &= \left(\frac{1}{3}x, \frac{1}{3}y + \frac{1}{3}\right),\\
    f_5((x, y)) &= \left(\frac{1}{3}x + \frac{2}{3}, \frac{1}{3}y + \frac{1}{3}\right),\\
    f_6((x, y)) &= \left(\frac{1}{3}x, \frac{1}{3}y + \frac{2}{3}\right),\\
    f_7((x, y)) &= \left(\frac{1}{3}x + \frac{1}{3}, \frac{1}{3}y  + \frac{2}{3}\right),\\
    f_8((x, y)) &= \left(\frac{1}{3}x + \frac{2}{3}, \frac{1}{3}y  + \frac{2}{3}\right).
\end{align*}

Denote \(f(X) = \bigcup_{i = 1}^{8} f_i(X)\), \(\square_0 = [0, 1]^2\) and \(\square_k = f^{(k)}(\square_0)\), and the Sierpinski carpet as \(\square\).

The Sierpinski carpet's IFS satisfies the OSC: notice that the open unit square \(U = (0, 1)^2\) satisfies the OSC by
\begin{align*}
    f_1(U) &= \left(0, \frac{1}{3}\right) \times \left(0, \frac{1}{3}\right)\\
    f_2(U) &= \left(\frac{1}{3}, \frac{2}{3}\right) \times \left(0, \frac{1}{3}\right)\\
    f_3(U) &= \left(\frac{2}{3}, 1\right) \times \left(0, \frac{1}{3}\right)\\
    f_4(U) &= \left(0, \frac{1}{3}\right) \times \left(\frac{1}{3}, \frac{2}{3}\right)\\
    f_5(U) &= \left(\frac{2}{3}, 1\right) \times \left(\frac{1}{3}, \frac{2}{3}\right)\\
    f_6(U) &= \left(0, \frac{1}{3}\right) \times \left(\frac{2}{3}, 1\right)\\
    f_7(U) &= \left(\frac{1}{3}, \frac{2}{3}\right) \times \left(\frac{2}{3}, 1\right)\\
    f_8(U) &= \left(\frac{2}{3}, 1\right) \times \left(\frac{2}{3}, 1\right)\\.
\end{align*}
and notice they are all disjoint subsets of \(U\). Therefore, the dimension of \(\square\), \(s\), must satisfy that
\[
8 \cdot 3^{-s} = 1 \implies s = \frac{\log 8}{\log 3}.
\]

Let \(\square\) be the Sierpinski carpet, and denote \(S_k\) as a square generated in the \(k\)-th iteration of the process, for example, \(S_0 = (0, 1)^2\), note that \(S_k\) has side length \(3^{-k}\) and there are \(8^{k}\) such squares in the process. Now, we define the measure \(\mu\):
\[
\mu(S_k \cap \square) = 8^{-k},
\]
and for any \(U \subset \square\) a subset of the Sierpinski carpet, we can approximate it using some smaller squares:
\[
\mu(U) = \inf\left\{\sum_{i = 1}^{+\infty} \mu(E_i): U \subset \bigcup_{i = 1}^{+\infty} E_i, E_i \text{ is some square in the process}\right\}.
\]

Assume that \(|U| < \sqrt{1}\). There must therefore exist some \(k\) such that
\(3^{-(k + 1)} \leq |U| < 3^{-k}\) for some integer \(k\). Then, \(U \subset \left({S_k}_1 \cup {S_k}_2 \cup {S_k}_3 \cup {S_k}_4\right) \cap \square\) where \({S_k}_i\) is one of the squares in the \(k\)-th step of the construction, i.e., \(U\) can only be situated in at most 4 squares at the same time. Therefore,
\begin{align*}
    \mu(U) &\leq 4 \cdot 8^{-k}\\
    &= 4 \cdot \left(3^s\right)^{-k}\\
    &= 4 \cdot \left(3^{-k}\right)^s\\
    &= 4 \cdot 3^s \cdot \left(3^{-k - 1}\right)^s\\
    &\leq 4 \cdot 8 \cdot |U|^s\\
    &= 32 \cdot |U|^s.
\end{align*}

This means the hypothesis for the mass distribution principle is true for \(\epsilon = 1\) and \(c = 32\). Therefore, from mass distribution principle, we have
\[
\h^s(\square) \geq \frac{1}{32}.
\]