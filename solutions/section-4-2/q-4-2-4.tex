Consider the set \(V = \left(-\frac{1}{4}, \frac{5}{4}\right) = \left(-\frac{3}{12}, \frac{15}{12}\right)\). For the IFS \((f_1, f_2)\) for the Cantor set \(C\), we have
\begin{align*}
    f_1(x) &= \frac{x}{3},\\
    f_2(x) &= \frac{x}{3} + \frac{2}{3}.
\end{align*}

Therefore, we have
\begin{align*}
    f_1(V) &= \left(-\frac{1}{12}, \frac{5}{12}\right),\\
    f_2(V) &= \left(\frac{9}{12}, \frac{13}{12}\right).
\end{align*}

Apparently \(f_1(V) \subset V, f_2(V) \subset V, f_1(V) \cap f_2(V) = \emptyset\). \(V \supset [0, 1] \supset C\) so not only does \(V\) satisfy the strong open set condition that \(C \cap V = C \neq \emptyset\), \(V\) also satisfies that \(V \supset C\) that it contains \(C\) completely.

For the Sierpinski Carpet, it is possible for \(V = (0, 1)^2\) to satisfy the SOSC. Recall that the IFS for the Sierpinski Carpet satisfies that

\begin{align*}
    g_1((x, y)) &= \left(\frac{x}{3}, \frac{y}{3}\right),\\
    g_2((x, y)) &= \left(\frac{x}{3}, \frac{y}{3} + \frac{1}{3}\right),\\
    g_3((x, y)) &= \left(\frac{x}{3}, \frac{y}{3} + \frac{2}{3}\right),\\
    g_4((x, y)) &= \left(\frac{x}{3} + \frac{1}{3}, \frac{y}{3}\right),\\
    g_5((x, y)) &= \left(\frac{x}{3} + \frac{1}{3}, \frac{y}{3} + \frac{2}{3}\right),\\
    g_6((x, y)) &= \left(\frac{x}{3} + \frac{2}{3}, \frac{y}{3}\right),\\
    g_7((x, y)) &= \left(\frac{x}{3} + \frac{2}{3}, \frac{y}{3} + \frac{1}{3}\right),\\
    g_8((x, y)) &= \left(\frac{x}{3} + \frac{2}{3}, \frac{y}{3} + \frac{2}{3}\right),\\
\end{align*}
and therefore
\begin{align*}
    g_1(V) &= \left(0, \frac{1}{3}\right) \times \left(0, \frac{1}{3}\right),\\
    g_2(V) &= \left(0, \frac{1}{3}\right) \times \left(\frac{1}{3}, \frac{2}{3}\right),\\
    g_3(V) &= \left(0, \frac{1}{3}\right) \times \left(\frac{2}{3}, 1\right),\\
    g_4(V) &= \left(\frac{1}{3}, \frac{2}{3}\right) \times \left(0, \frac{1}{3}\right),\\
    g_5(V) &= \left(\frac{1}{3}, \frac{2}{3}\right) \times \left(\frac{2}{3}, 1\right),\\
    g_6(V) &= \left(\frac{2}{3}, 1\right) \times \left(0, \frac{1}{3}\right),\\
    g_7(V) &= \left(\frac{2}{3}, 1\right) \times \left(\frac{1}{3}, \frac{2}{3}\right),\\
    g_8(V) &= \left(\frac{2}{3}, 1\right) \times \left(\frac{2}{3}, 1\right),\\
\end{align*}
and we can see they are pairwise disjoint and all subsets of \(V\). Furthermore, let \(g(E) = \bigcup_{i = 1}^{8} g_i(E)\), note that the point \(\left(\frac{1}{3}, \frac{1}{3}\right) \in V\), and also note that
\[
g_8((1, 1)) = (1, 1) \implies (1, 1) \in g^{(k)}([0, 1]^2),
\]
\[
g_1((1, 1)) = \left(\frac{1}{3}, \frac{1}{3}\right),
\]
and therefore
\[
\left(\frac{1}{3}, \frac{1}{3} \in g^{(k)}([0, 1]^2)\right),
\]
and therefore \(\left(\frac{1}{3}, \frac{1}{3}\right) \in \square\), the Sierpinski Carpet. But \(\left(\frac{1}{3}, \frac{1}{3}\right) \in V\) as well, so \(V \cap C \supset \left\{\left(\frac{1}{3}, \frac{1}{3}\right)\right\}\), and therefore \(V \cap C \neq \emptyset\).

However, it is impossible for such \(V \supset C\) to satisfy the SOSC. This is because if \(V \supset C\), then \((0, 1) \in V, (1, 1) \in V\). But notice
\[
g_4((0, 1)) = \left(\frac{1}{3}, \frac{1}{3}\right) = g_1((1, 1)),
\]
and therefore \(g_1(V) \cap g_4(V) \supset \left\{\left(\frac{1}{3}, \frac{1}{3}\right)\right\}\), and \(g_1(V) \cap g_4(V) \neq \emptyset\) which violates the pairwise disjoint condition of the OSC, hence the second part is impossible for the Sierpinski carpet.