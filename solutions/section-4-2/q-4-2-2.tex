We alter slightly the definition of the natural measure. Specifically, we assign an uneven amount of measure on the left interval created and the right interval created each step. Let \(0 < p < 1\) such that \(p \cdot 3^s > 1\) where \(s = \log 2 / \log 3\), i.e., \(p > \frac{1}{2}\). Let the IFS of the Cantor set be \((f_1, f_2)\) where \(f_1(x) = \frac{1}{3}x\) and \(f_2(x) = \frac{1}{3} x + \frac{2}{3}\), and we denote \(f_{i_1, i_2, \ldots, i_m} (E)\) to be \(f_{i_1}(f_{i_2}(\ldots(f_{i_m}(E))))\), where each \(i_k \in \{1, 2\}\). Let \(t_1 = p, t_2 = 1 - p\), and define the measure \(\mu\) such that
\[
\mu\left(f_{i_1, i_2, \ldots, i_m} (C)\right) = \mu\left(f_{i_1, i_2, \ldots, i_m} ([0, 1]) \cap C\right) = \prod_{j = 1}^{m} t_{i_j} = t_{i_1} t_{i_2} \ldots t_{i_m}.
\]

This means every time a left interval is created, it takes up \(p\) of the measure of the original interval, with the right interval taking up \(1 - p\) of the measure.

Now, we can generalise this for any \(U \subset C\) simply using the similar concept as before, by approximating it with smaller and smaller intervals \(\cap C\):
\[
\mu(U) = \inf\left\{\sum_{i = 1}^{\mu(E_i)} : U \subset \bigcup_{i} E_i, E_i \text{ is an interval in the construction}\right\}.
\]

It is not difficult to see the non-negativity and the measure of the empty set being zero from the definition. The countable additivity is also true: consider disjoint Borel subsets \(\{E_i\}\) of \(C\), we will have \(\mu (\bigcup E_i) = \sum \mu(E_i)\) since the intervals are either disjoint or contains one-another (and we obviously don't want them to contain one-another to get closer to the \(\inf\) since simply removing the smaller one won't change the validity of the cover), and so they are disjoint, and so this holds.

Now consider \(\frac{\mu(U)}{|U|^s}\) for \(U = f_{1, 1, \ldots, 1}(C)\) where there are \(n\) 1s. It's not difficult to see that
\[
|U| = \left(\frac{1}{3}\right)^n,
\]
and that
\[
\mu(U) = p^n.
\]

Notice that
\[
\frac{\mu(U)}{|U|^s} = \frac{p^n}{3^{-sn}} = (2p)^n.
\]

Notice that this value decreases as \(p\) increases. For any \(\epsilon > 0\), we can find \(n\) that is sufficiently big enough such that \(|U| = 3^{-n} > \epsilon\), and for any \(c > 0\), we can further increase \(n\) to be big enough such that \(\mu(U) / |U|^s = (2p)^n > c\), since \(2p > 1\) and \((2p)^n\) is increasing with respect to \(n\). This means that for all \(c, \epsilon > 0\), we can find some \(U \subset C, |U| \leq \epsilon\), such that \(\mu (U) > c|U|^s\) for this measure \(\mu\) that we just constructed, as desired.