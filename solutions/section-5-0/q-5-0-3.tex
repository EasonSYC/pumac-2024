I will prove that \(a_i \geq a_{i+1}\) -- if I could show that for any possible value of \(a_i\), there exists a covering of \(T_{n+1}\) with the same value, then we would be done (as \(a_{i+1}\) is defined as the minimum of these values, there could be one less than these but it is at most \(a_{i}\)).

However this is clear as if we have a covering of \(T_n\), we can scale it down by scale factor \(\frac{1}{2}\) centred at \((0,0)\) to cover one of the three largest triangles of \(T_{n+1}\). This is clearly a covering as these triangles have been scaled down in a similar fashion to the first contraction function in the IFS, thus it's the same as just adding this function to the beginning of \(S_{I_n}(\Delta)\). This will have half the diameter, and have \(\frac{1}{3}\) of the measure (as there are 3 identical largest triangles that could be filled). Thus letting D be the diameter, and M be the measure, our value for \(a_{i+1}\) is \(\frac{(D/2)^s}{M/3} = \frac{D^s}{M}\), which is the same value as our covering for \(a_i\). Thus our lemma is proven.