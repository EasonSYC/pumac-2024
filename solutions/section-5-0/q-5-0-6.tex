We show that \(a_3\) equals \(0.910411\) -- precisely
\[\frac{7^{\frac{\log(3)}{\log(2)}}}{24}.\]

This is done using this program, which brute-forces every combination of triangles to make the cover, finds the diameter, and computes the value of \(a_3\). Then it finds the minimum out of all these values and prints it out. This is extremely naive and would not work for \(a_4\).


\begin{verbatim}
#include <cmath>
#include <vector>
#include <set>
#include <tuple>
#include <algorithm>
#include <iostream>
#include <limits>

using namespace std;
using Point = pair<double, double>;
using Triangle = vector<Point>;

double root3 = sqrt(3.0);
double s = log(3.0) / log(2.0);

Point div(const Point& pt, double x) {
    return {pt.first / x, pt.second / x};
}

Point add(const Point& pt1, const Point& pt2) {
    return {pt1.first + pt2.first, pt1.second + pt2.second};
}

Point f0(const Point& pt) {
    return div(pt, 2.0);
}

Point f1(const Point& pt) {
    return add({0.5, 0.0}, div(pt, 2.0));
}

Point f2(const Point& pt) {
    return add({0.25, root3 / 4.0}, div(pt, 2.0));
}

using Func = Point (*)(const Point&);

vector<Func> funcs = {f0, f1, f2};

Triangle f(int n, const Triangle& tri) {
    Triangle result;
    for (const auto& pt : tri) {
        result.push_back(funcs[n](pt));
    }
    return result;
}

vector<Triangle> nexttri(const vector<Triangle>& tri) {
    vector<Triangle> tri1;
    for (const auto& t : tri) {
        for (int i = 0; i < 3; ++i) {
            tri1.push_back(f(i, t));
        }
    }
    return tri1;
}

set<Point> nextpts(const set<Point>& pts) {
    set<Point> pts1;
    for (const auto& pt : pts) {
        for (int i = 0; i < 3; ++i) {
            pts1.insert(funcs[i](pt));
        }
    }
    return pts1;
}

double dist(const Point& pt1, const Point& pt2) {
    return sqrt(pow(pt1.first - pt2.first, 2)
            + pow(pt1.second - pt2.second, 2));
}

double diameter(const set<Point>& pts) {
    double diam = 0.0;
    for (const auto& pt1 : pts) {
        for (const auto& pt2 : pts) {
            diam = max(diam, dist(pt1, pt2));
        }
    }
    return diam;
}

int main() {
    vector<Triangle> tri0 = {{{{0, 0}, {1, 0}, {0.5, root3 / 2}}}};
    set<Point> pts0 = {{0, 0}, {1, 0}, {0.5, root3 / 2}};

    auto tri1 = nexttri(tri0);
    auto pts1 = nextpts(pts0);
    int N = 3;

    for (int i=0;i<N-1;++i){
        tri1 = nexttri(tri1);
        pts1 = nextpts(pts1);
    }



    double mina = numeric_limits<double>::max();

    size_t numTriangles = tri1.size();

    double dd = -1, nn = -1; // store exact values
    for (size_t i = 1; i < (1u << numTriangles); ++i) {
        
        vector<Triangle> tri;
        set<Point> pts;

        for (size_t j = 0; j < numTriangles; ++j) {
            if (i & (1u << j)) {
                tri.push_back(tri1[j]);
                for (const auto& pt : tri1[j]) {
                    pts.insert(pt);
                }
            }
        }

        double d = diameter(pts);
        double n = static_cast<double>(tri.size());
        double a = pow(d, s) / (n / pow(3.0, N));
        if (a < mina) {
            mina = a;
            dd = d;
            nn = n;
        }
    }

    cout << mina << endl;
    cout << dd << " " << nn << endl;
    return 0;
}
\end{verbatim}
