We first prove that \(C\) is closed if and only if it contains all points \(x\) such that for all \(r > 0\), \(B_x(r) \cap C \neq \emptyset\).

\begin{itemize}
    \item \textbf{Only if direction.} We would like to prove that if \(C\) is closed, then for \(x\) s.t. for all \(r > 0\), \(B_x(r) \cap C \neq \emptyset\), we have \(x \in C\).

    Since \(C\) is closed, \(\R^n \setminus C\) is open. B.W.O.C. suppose that \(x \notin C\), so \(x \in \R^n \setminus C\). By the definition of an open set being a union of open balls, \(x\) must be within some open ball (say \(B_p(s)\) which is itself a subset of \(\R^n \setminus C\). It is not difficult to see that \(s > |x - p|\) by definition of an open ball, and from the triangular inequality that \(B_x(s - |x - p|) \subset B_p(s) \subset \R^n \setminus C\): for some point \(y\) s.t. \(|y - x| < s - |x - p|\),
    \[|y - p| \leq |y - x| + |x - p| < s - |x - p| + |x - p| < s.\]

    Therefore for \(r = s - |x - p|\), \(B_x(r) \subset \R^n \setminus C\), and \(B_x(r) \cap C = \emptyset\), which contradicts with given that \(B_x(r) \cap C \neq \emptyset\).

    \item \textbf{If direction.} We would like to prove that if \(x\) is s.t. for all \(r > 0\), \(B_x(r) \cap C \neq \emptyset\), then \(x \in C\), then \(C\) is closed.

    To show that \(C\) is closed, we show that \(\R^n \setminus C\) is open. From the given condition, we know that for all \(x \in \R^n \setminus C\), there exists some \(r_x > 0\) s.t. \(B_x(r_x) \subset \R^n \setminus C\).
    
    (If not, then for all \(r > 0\), \(B_x(r_x) \not\subset \R^n \setminus C\), which implies \(B_x(r_x) \cap C \neq \emptyset\) which implies \(x \in C\) which contradicts with \(x \in \R^n \setminus C\).)
    
    Define that
    \[
        S = \bigcup_{x \in \R^n \setminus C} B_x (r_x).
    \]
    Obviously \(S \subset \R^n \setminus C\) by how we defined \(r_x\). At the same time \(\R^n\setminus C \subset S\) since \(x \in B_x(r_x) \subset S\). Therefore, \(S = \R^n \setminus C\) and we have shown that it is a union of open balls, therefore open, and therefore \(C\) being closed.
\end{itemize}

Now we show the original statement. Assume \(F_1, F_2\) are both attractors of an IFS \((f_1, \ldots, f_m)\) and \(F_1, F_2\) are non-empty and compact. We would like to show that \(F_1 = F_2\). Define \(f(X) = \bigcup_i f_i(X)\), we have \(f(F_1) = F_1, f(F_2) = F_2\).

Let \(\mathcal{C}\) be the set of all non-empty compact subsets of \(\R^n\). We consider defining a metric \(d\) on \(\mathcal{C}\) for \(A, B \in \mathcal{C} \subset \R^n\) by follows.

We first define the distance of a point \(x \in X\) to the set \(Y\) as
\[
d(x, Y) = \inf_{y \in Y} |x - y|,
\]
which is the 'shortest distance from \(x\) to any point \(y \in Y\)'. Now, the distance between two sets (to let it satisfy properties of metrics), could then be defined as follows:
\[
d(A, B) = \max\left\{\sup_{a \in A} d(a, B), \sup_{b \in B} d(b, A) \right\},
\]
as 'the furthest the shortest distance can be'.

It is a metric since for \(A, B, C \in \mathcal{C}\):
\begin{itemize}
    \item \(d(A, B) \geq 0\) is true since \(|x - y| \geq 0\).
    
    If \(d(A, B) = 0\), then it must be true that \(\sup_{a \in A} d(a, B) = \sup_{b \in B} d(b, A) = 0\), which means \(\inf_{b \in B} |a - b| = 0\) for all \(a \in A\) and \(\inf_{a \in A} |a - b| = 0\) for all \(b \in B\).

    This means that for all \(a \in A\), for any \(\epsilon > 0\), there exists a \(b \in B\) such that \(a \in B_b(\epsilon)\). So \(b\) satisfies that for all \(\epsilon > 0\), \(B_b(\epsilon) \cap A \supset \{a\}\) and therefore \(B_b(\epsilon) \cap A \neq \emptyset\). But since \(A\) is compact, it is closed, and therefore, it contains all points \(x\) such that for \(r > 0, B_x(r) \cap C \neq \emptyset\), and therefore \(b \in A\) for all \(b \in B\), therefore \(B \subset A\). A symmetrical argument can be made to show \(A \subset B\) and therefore \(A = B\).
    
    \item \(d(A, B) = d(B, A)\) is true by symmetry of the definition;
    \item \(d(A, B) + d(B, C) \geq d(A, C)\) inherits from the fact that \(|a - b|\) is a metric. Specifically, \(d(a, C) \leq |a - c| \leq |a - b| + |b - c|\) by definition and by triangular inequality for any \(a \in A, b \in B, c \in C\). Taking \(\inf_{c \in C}\) will therefore then give us \(d(a, C) \leq |a - b| + d(B, C)\), for which if we take \(\inf_{b \in B}\) will therefore be \(d(a, C) \leq d(a, B) + d(B, C)\), but since \(d(a, B) \leq d(A, B)\) we must have \(d(a, C) \leq d(A, B) + d(B, C)\), and therefore \(\sup_{a \in A} d(a, C) \leq d(A, B) + d(B, C)\). A similar argument can be shown on \(\sup_{c \in C} d(A, c) \leq d(A, B) + d(B, C)\), and therefore
    \[d(A, C) = \max\left\{\sup_{a \in A} d(a, C), \sup_{c \in C} d(b, A) \right\}  \leq d(A, B) + d(B, C)\]
    as required.
\end{itemize}

Now, we show that \(f\) has only one stationary point on \(\mathcal{C}\). By definition, for any \(A, B \in \mathcal{C}\), we must have
\[
d(f(A), f(B)) = d\left(\bigcup_{i = 1}^{m} f_i(A), \bigcup_{i = 1}^{m} f_i(B)\right),
\]

Let
\[M = \max_{1 \leq i \leq m} d(f_i(A), f_i(B)),\]
we must have
\[\sup_{a \in f_i(A)} d(a, f_i(B)) \leq M\]
for all \(1 \leq i \leq M\) by definition, and therefore that
\[\sup_{a \in \bigcup_{i = 1}^{m} f_i(A)} d\left(a, \bigcup_{i = 1}^{m} f_i(B)\right) \leq M,\]
and similarly
\[\sup_{a \in \bigcup_{i = 1}^{m} f_i(B)} d\left(a, \bigcup_{i = 1}^{m} f_i(A)\right) \leq M.\]

This then shows that
\[
d(f(A), f(B)) \leq M = \max_{1 \leq i \leq m} d(f_i(A), f_i(B)) \leq \left(\max_{1 \leq i \leq m} c_i\right)d(A, B).
\]

Therefore, if \(F_1, F_2 \in \mathcal{C}\) are bot attractors for the IFS \(\left(f_1, f_2, \ldots, f_m\right)\), i.e., \(f(F_1) = F_1, f(F_2) = F_2\), we must have
\[
d(f(F_1), f(F_2)) \leq d(F_1, F_2) \leq \left(\max_{1 \leq i \leq m} c_i\right) d(F_1, F_2)
\]
since we must have \(0 < \max_{1 \leq i \leq m} c_i < 1\), this implies that we must have \(d(F_1, F_2) = 0\), and therefore, \(F_1 = F_2\), which shows the non-empty compact attractor is unique, as desired.