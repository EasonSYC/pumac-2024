For this question, due to the constructive proof of the existence of a attractor for any IFS, we just have to show that the IFS's union process is equivalent to the iterative process of the iterative definition of the fractal, and that the starting point is compact.

\begin{itemize}
    \item \textbf{Sierpinski Carpet.} An equivalent definition of the Sierpinski Carpet is by starting with \([0, 1] \times [0, 1]\), and each time shrinking the size of the whole shape with scale factor \(\frac{1}{3}\) and creating 8 copies of that. Therefore, we can define the IFS \((f_1, \ldots, f_8)\) on \(\R^2 \to \R^2\):
    \begin{align*}
        f_1((x, y)) &= \left(\frac{1}{3}x, \frac{1}{3}y\right),\\
        f_2((x, y)) &= \left(\frac{1}{3}x + \frac{1}{3}, \frac{1}{3}y\right),\\
        f_3((x, y)) &= \left(\frac{1}{3}x + \frac{2}{3}, \frac{1}{3}y\right),\\
        f_4((x, y)) &= \left(\frac{1}{3}x, \frac{1}{3}y + \frac{1}{3}\right),\\
        f_5((x, y)) &= \left(\frac{1}{3}x + \frac{2}{3}, \frac{1}{3}y + \frac{1}{3}\right),\\
        f_6((x, y)) &= \left(\frac{1}{3}x, \frac{1}{3}y + \frac{2}{3}\right),\\
        f_7((x, y)) &= \left(\frac{1}{3}x + \frac{1}{3}, \frac{1}{3}y  + \frac{2}{3}\right),\\
        f_8((x, y)) &= \left(\frac{1}{3}x + \frac{2}{3}, \frac{1}{3}y  + \frac{2}{3}\right).
    \end{align*}

    Here, \(f_1\) to \(f_8\) generates the bottom-left, bottom, bottom-right, left, right, top-left, top, top-right \(\frac{1}{3}\) by \(\frac{1}{3}\) respectively. We can verify this simply by plugging two values for the first iteration (say, \((0, 0)\) and \((1, 0)\)) since they are affine. This is an IFS since \(f_1\) to \(f_8\) are all shrinking, in fact with contraction ratio \(\frac{1}{3}\)
    
    It is not difficult to see that \(f(X) = \bigcup_{i = 1}^{8} f_i(X)\) maps \(\square_k\) to \(\square_{k + 1}\) for \(k \in \N\) since this is exactly the definition. Therefore, the attractor of this IFS is
    \[
        \square = \bigcap_{i \in \N} f^{(n)}(\square_0)
    \]
    since \(\square_0 = [0, 1]^2\) is compact. This is exactly the Sierpinski Carpet \(\square\).

    \item \textbf{Koch Curve.} We can see that the iterative process of the Koch curve can be split into four pieces: the two pieces that got shrunken with scale factor \(\frac{1}{3}\) and gets copied four times, two of which are then rotated into the correct position. Therefore, define the IFS \((f_1, f_2, f_3, f_4)\) on \(\R^2 \to \R^2\):
    \begin{align*}
        f_1((x, y)) &= \left(\frac{1}{3}x, \frac{1}{3}y\right),\\
        f_2((x, y)) &= \left(\frac{1}{6}x - \frac{\sqrt{3}}{6}y + \frac{1}{3}, \frac{\sqrt{3}}{6}x + \frac{1}{6}y\right),\\
        f_3((x, y)) &= \left(\frac{1}{6}x + \frac{\sqrt{3}}{6}y + \frac{1}{2}, -\frac{\sqrt{3}}{6}x + \frac{1}{6}y + \frac{\sqrt{3}}{6}\right),\\
        f_4((x, y)) &= \left(\frac{1}{3}x + \frac{2}{3}, \frac{1}{3}y\right).
    \end{align*}

    Here, \(f_1\) generates the left part, \(f_4\) generates the right part, \(f_2\) generates the part tilted counter clockwise by \(\frac{\pi}{3}\), and \(f_3\) generates the part tilted clockwise by \(\frac{\pi}{3}\). This is an IFS since \(f_1\) to \(f_4\) are all shrinking, in fact with contraction ratio \(\frac{1}{3}\).

    It is not difficult to see that \(f(X) = \bigcup_{i = 1}^{4} f_i(X)\) maps \(K_k\) to \(K_{k + 1}\) for \(k \in \N\) since this is exactly the definition. Therefore, the attractor of this IFS is
    \[
        K = \bigcap_{i \in \N} f^{(n)}(K_0)
    \]
    since \(K_0 = [0, 1]\) is compact. This is exactly the Koch Curve \(K\).

    \item \textbf{Minkowski Sausage.} We can see that the iterative process of the Minkowski Sausage can be split into 8 pieces, all shrunken with scale factor \(\frac{1}{4}\) and affine transformed into the correct position. Therefore, define the IFS \((f_1, \ldots, f_8)\) on \(\R^2 \to \R^2\):
    \begin{align*}
        f_1((x, y)) &= \left(\frac{1}{4}x, \frac{1}{4}y\right),\\
        f_2((x, y)) &= \left(-\frac{1}{4}y + \frac{1}{4}, \frac{1}{4}x\right),\\
        f_3((x, y)) &= \left(\frac{1}{4}x + \frac{1}{4}, \frac{1}{4}y + \frac{1}{4}\right),\\
        f_4((x, y)) &= \left(\frac{1}{4}y + \frac{1}{2}, -\frac{1}{4}x + \frac{1}{4}\right),\\
        f_5((x, y)) &= \left(\frac{1}{4}y + \frac{1}{2}, -\frac{1}{4}x - \frac{1}{4}\right),\\
        f_6((x, y)) &= \left(\frac{1}{4}x + \frac{1}{2}, \frac{1}{4}y - \frac{1}{4}\right),\\
        f_7((x, y)) &= \left(-\frac{1}{4}y + \frac{3}{4}, \frac{1}{4}x - \frac{1}{4}\right),\\
        f_8((x, y)) &= \left(\frac{1}{4}x + \frac{3}{4}, \frac{1}{4}y\right).\\
    \end{align*}

    Here, \(f_1, f_3, f_6, f_8\) generates the four horizontal parts, \(f_2, f_7\) generates the two counter-clockwise rotations, and \(f_4, f_5\) generates the two clockwise rotations. This is an IFS since \(f_1\) to \(f_8\) are all shrinking, in fact with contraction ratio \(\frac{1}{4}\).

    It is not difficult to see that \(f(X) = \bigcup_{i = 1}^{8} f_i(X)\) maps \(M_k\) to \(M_{k + 1}\) for \(k \in \N\) since this is exactly the definition. Therefore, the attractor of this IFS is
    \[
        M = \bigcap_{i \in \N} f^{(n)}(M_0)
    \]
    since \(M_0 = [0, 1]\) is compact. This is exactly the Minkowski Triangle \(M\).
\end{itemize}