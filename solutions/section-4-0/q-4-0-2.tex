To show that \(f(K)\) is compact given \(f\) is a contraction and \(K\) is compact, we show that \(f(K)\) is closed and bounded.

\begin{itemize}
    \item \textbf{\(f(K)\) is bounded.} Since \(K\) is bounded, there exists some \(r > 0\) such that \(K \subset B_{(0, \ldots, 0)} (r)\). In other words, for all \(x \in K\), \(|x - (0, \ldots, 0)| < r\) .
        
    Since \(f\) is a contraction, set \(y = (0, \ldots, 0)\), we will see that \(|f(x) - f((0, \ldots, 0))| = c|x - (0, \ldots, 0)| < cr\).
    
    By the triangular inequality, we have
    \begin{align*}
        |f(x) - (0, \ldots, 0)| &\leq |f(x) - f((0, \ldots, 0))| + |f((0, \ldots, 0)) - (0, \ldots, 0)|\\
        &< cr + |f((0, \ldots, 0)) - (0, \ldots, 0)|.
    \end{align*}
    
    Therefore, \(f(K) \subset B_{(0, \ldots, 0)} (cr + |f((0, \ldots, 0)) - (0, \ldots, 0)|)\), which shows \(f(K)\) is bounded.
        
    \item \textbf{\(f(K)\) is closed.} We would like to first prove the lemma that a set is closed if and only if all its limit points (a point that can be reached using a limit of a sequence of points within the set) is contained in the set.
    \begin{itemize}
        \item \textbf{If direction.} Suppose a set \(C\) contains all its limit points. Now suppose \(x \in \mathbb{R}^n \setminus C\). Then \(x\) must not be a limit point of \(C\). By the \(\epsilon-\delta\) definition of a limit, it must be true that there is an open neighbourhood of \(x\) (an open ball centred at \(x\)) that does not intersect with \(C\). (It is not a limit point means it is not arbitrarily close to \(C\).) For every \(x \in \mathbb{R}^n \setminus C\), say such ball is \(B_x(r_x)\).

        Define that
        \[
            S = \bigcup_{x \in \R^n \setminus C} B_x (r_x).
        \]
        Obviously \(S \subset \R^n \setminus C\) by how we defined \(r_x\). At the same time \(\R^n\setminus C \subset S\) since \(x \in B_x(r_x) \subset S\). Therefore, \(S = \R^n \setminus C\) and we have shown that it is a union of open balls, therefore open, and therefore \(C\) being closed.
        
        \item \textbf{Only if direction.} Suppose \(C\) is closed. Then \(\mathbb{R}^n \setminus C\) is open. Suppose there is a limit point \(x \notin C\), and therefore \(x \in \mathbb{R}^n \setminus C\).

        By the definition of an open set being a union of open balls, \(x\) must be within some open ball (say \(B_p(s)\) which is itself a subset of \(\R^n \setminus C\). It is not difficult to see that \(s > |x - p|\) by definition of an open ball, and from the triangular inequality that \(B_x(s - |x - p|) \subset B_p(s) \subset \R^n \setminus C\): for some point \(y\) s.t. \(|y - x| < s - |x - p|\),
        \[|y - p| \leq |y - x| + |x - p| < s - |x - p| + |x - p| < s.\]
    
        Therefore for \(r = s - |x - p|\), \(B_x(r) \subset \R^n \setminus C\), and \(B_x(r) \cap C = \emptyset\), which contradicts with given that \(x\) is a limit point of \(C\) by considering \(\epsilon \leq r\) in the \(\epsilon-\delta\)-definition of a limit. Therefore \(C\) contains all its limit points.
    \end{itemize}

    We would also like to show that \(f\) is continuous. But this is straightforward since
    \[
        \lim_{x \to y} |f(x) - f(y)| = c |\lim_{x \to y} x - y| = 0
    \]
    which then implies that \(\lim_{x \to y} f(x) = f(y)\) which means \(f\) is continuous.

    Now we are ready to start the proof. We would like to show that \(f(K)\) is closed by showing it contains all its limit points. Consider a limit point formed by \(\{f(x_i)\}_{i = 1}^{+\infty}\) where \(x_i \in K\). We must have
    \[
        \lim_{i \to +\infty} f(x_i) = f\left(\lim_{i \to +\infty}x_i\right).
    \]

    Since \(K\) is closed, \(K\) contains all its limit points, and therefore \(\lim_{i \to +\infty}x_i \in K\), and therefore \(\lim_{i \to +\infty} f(x_i) = f\left(\lim_{i \to +\infty}x_i\right) \in f(K)\), which shows that \(f(K)\) contains all its limit points, hence closed.
\end{itemize}

We have shown that given \(K\) is closed, \(f(K)\) is closed. We have also shown that given \(K\) is bounded, \(f(K)\) is bounded. Therefore, given \(K\) is compact, \(f(K)\) is compact.